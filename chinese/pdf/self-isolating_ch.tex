%\documentclass[twocolumn,journal]{IEEEtran}
\documentclass[onecolumn,journal]{IEEEtran}
\usepackage{ctex}
\usepackage{amsfonts}
\usepackage{amsmath}
\usepackage{amsthm}
\usepackage{amssymb}
\usepackage{graphicx}
\usepackage[T1]{fontenc}
%\usepackage[english]{babel}
\usepackage{supertabular}
\usepackage{longtable}
\usepackage[usenames,dvipsnames]{color}
\usepackage{bbm}
%\usepackage{caption}
\usepackage{fancyhdr}
\usepackage{breqn}
\usepackage{fixltx2e}
\usepackage{capt-of}
%\usepackage{mdframed}
\setcounter{MaxMatrixCols}{10}
\usepackage{tikz}
\usetikzlibrary{matrix}
\usepackage{endnotes}
\usepackage{soul}
\usepackage{marginnote}
%\newtheorem{theorem}{Theorem}
\newtheorem{lemma}{Lemma}
%\newtheorem{remark}{Remark}
%\newtheorem{error}{\color{Red} Error}
\newtheorem{corollary}{Corollary}
\newtheorem{proposition}{Proposition}
\newtheorem{definition}{Definition}
\newcommand{\mathsym}[1]{}
\newcommand{\unicode}[1]{}
\newcommand{\dsum} {\displaystyle\sum}
\hyphenation{op-tical net-works semi-conduc-tor}
\usepackage{pdfpages}
\usepackage{enumitem}
\usepackage{multicol}
\usepackage[utf8]{inputenc}


\headsep = 5pt
\textheight = 730pt
%\headsep = 8pt %25pt
%\textheight = 720pt %674pt
%\usepackage{geometry}

\bibliographystyle{unsrt}

\usepackage{float}

 \usepackage{xcolor}

\usepackage[framemethod=TikZ]{mdframed}
%%%%%%%FRAME%%%%%%%%%%%
\usepackage[framemethod=TikZ]{mdframed}
\usepackage{framed}
    % \BeforeBeginEnvironment{mdframed}{\begin{minipage}{\linewidth}}
     %\AfterEndEnvironment{mdframed}{\end{minipage}\par}


%	%\mdfsetup{%
%	%skipabove=20pt,
%	nobreak=true,
%	   middlelinecolor=black,
%	   middlelinewidth=1pt,
%	   backgroundcolor=purple!10,
%	   roundcorner=1pt}

\mdfsetup{%
	outerlinewidth=1,skipabove=20pt,backgroundcolor=yellow!50, outerlinecolor=black,innertopmargin=0pt,splittopskip=\topskip,skipbelow=\baselineskip, skipabove=\baselineskip,ntheorem,roundcorner=5pt}

\mdtheorem[nobreak=true,outerlinewidth=1,%leftmargin=40,rightmargin=40,
backgroundcolor=yellow!50, outerlinecolor=black,innertopmargin=0pt,splittopskip=\topskip,skipbelow=\baselineskip, skipabove=\baselineskip,ntheorem,roundcorner=5pt,font=\itshape]{result}{Result}


\mdtheorem[nobreak=true,outerlinewidth=1,%leftmargin=40,rightmargin=40,
backgroundcolor=yellow!50, outerlinecolor=black,innertopmargin=0pt,splittopskip=\topskip,skipbelow=\baselineskip, skipabove=\baselineskip,ntheorem,roundcorner=5pt,font=\itshape]{theorem}{Theorem}

\mdtheorem[nobreak=true,outerlinewidth=1,%leftmargin=40,rightmargin=40,
backgroundcolor=gray!10, outerlinecolor=black,innertopmargin=0pt,splittopskip=\topskip,skipbelow=\baselineskip, skipabove=\baselineskip,ntheorem,roundcorner=5pt,font=\itshape]{remark}{Remark}

\mdtheorem[nobreak=true,outerlinewidth=1,%leftmargin=40,rightmargin=40,
backgroundcolor=pink!30, outerlinecolor=black,innertopmargin=0pt,splittopskip=\topskip,skipbelow=\baselineskip, skipabove=\baselineskip,ntheorem,roundcorner=5pt,font=\itshape]{quaestio}{Quaestio}

\mdtheorem[nobreak=true,outerlinewidth=1,%leftmargin=40,rightmargin=40,
backgroundcolor=yellow!50, outerlinecolor=black,innertopmargin=5pt,splittopskip=\topskip,skipbelow=\baselineskip, skipabove=\baselineskip,ntheorem,roundcorner=5pt,font=\itshape]{background}{Background}

%TRYING TO INCLUDE Ppls IN TOC
\usepackage{hyperref}


\begin{document}
\title{\color{Brown} 自我隔离指南
 \\
\vspace{-0.35ex}}
\author{Chen Shen 和 Yaneer Bar-Yam \\ 新英格兰复杂系统研究所 \\
\vspace{+0.35ex}
\small{\textit({翻译:Yanxia Fei  校对:David Xie})}\\
 \number 2020 年 \number 3 月 \number 15 日
  \vspace{-14ex} \\


\bigskip
\bigskip

\textbf{}
 }

\maketitle


\flushbottom % Makes all text pages the same height

%\maketitle % Print the title and abstract box

%\tableofcontents % Print the contents section

\thispagestyle{empty} % Removes page numbering from the first page

%----------------------------------------------------------------------------------------
%	ARTICLE CONTENTS
%----------------------------------------------------------------------------------------

%\section*{Introduction} % The \section*{} command stops section numbering

%\addcontentsline{toc}{section}{\hspace*{-\tocsep}Introduction} % Adds this section to the table of contents with negative horizontal space equal to the indent for the numbered sections

%\tableofcontents
%\section{ Introduction}
\renewcommand{\thefootnote}{\fnsymbol{footnote}}




\begin{multicols}{2}

本指南适用于 COVID-19 核酸检测结果呈阳性但症状轻微或无症状的患者。轻微症状包括低热、轻微乏力、干咳,但无肺炎表现且无并发慢性疾病。
如果当地医疗资源紧张,无法为出现中度或更严重症状的患者安排住院治疗,本指南也可以提供指导。请牢记,此处提供的建议只针对轻型或无症状患者。
一旦出现呼吸困难、高热等症状,请立即就医。

\section*{总方针}

\begin{itemize}

  \item 大多数患者都可以完全恢复健康。隔离是临时的必需要求,通常需要隔离 14 天。
  \item \textbf{强烈建议单独居住,至少要有独立的房间。}

\end{itemize}


\section*{如果您单独居住}

\begin{itemize}

  \item 密切跟踪您的健康状况。请在明显的地方清晰地记录下自己的健康日志。日志中应该包括记录的日期和时间、心率、血氧饱和度(通过血氧仪测量)、任何症状、食物、服用的药物和剂量。
  \item 及时了解当地为症状加重患者提供的的应对措施。在快速拨号中设置好关键联系人。(至少每天)都要与家人或朋友通报状况。告知他们所有自己的紧急联系信息,以及考虑到万一您失去活动能力时进入您家所需的任何信息。
  \item 多饮水可保护您的健康,均衡饮食,规律作息。保持或开发出或有趣或有教育意义的活动,比如阅读,玩单机游戏或在线游戏,以及其他的线上互动。
  \item 经常用肥皂或洗手液洗手,每次不少于20秒。
  \item 定期给居住空间通风至关重要。
  \item 经常清洗床单、毛巾和衣服。把您与他人的衣物分开洗涤。
  \item 避免与您的宠物以及其他动物接触。做不到的话,请确保在接触宠物前后都要佩戴口罩并洗手。
  \item 避免与他人进行身体接触,但可以通过短信、电话、视频聊天或其他电子途径与家人和朋友保持联系。这在很多方面都很重要,可以帮助人们保持积极的愿景。
  \item 如果有关疫情爆发的新闻使您焦虑,可以尽量不要过多地加以关注,避免进一步加重精神健康负担。
  \item 维持较为恒定的日常生活习惯。可能的话请保持适度或者少量的体育锻炼。
  \item 与家人朋友或地方当局协调如何解决每日膳食等后勤保障工作。追踪所有必需品的使用情况,注意每一种必需品即告短缺的时间,预先安排采购计划并提前通知供应商。最好采用无接触递送。与快递员接触时请佩戴口罩和手套。
  \item 向地方政府和医疗机构咨询自我隔离的时长以及结束自我隔离的条件。离开隔离区前请系统地进行清理、清洗和洗涤衣物。在此之后仍然要小心。
  \item 在郊区或农村地区,有些房屋可以做到避免与他人接触或共享空间而单独进出。在这些地方,避免他人单独进行散步也是可以的。请不要忘记您在隔离,不要与他人互动,也不要进入在您逗留时或之后可能有其他人进入的空间。

\end{itemize}

\section*{如果您与其他人共同居住}

\begin{itemize}

  \item 与您同住的人会被视为“密切接触者”,应该遵循当地为密切接触者提供的行动指南,比如避免与他人不必要的接触。
  \item 住所中的其他居民应该避免接待访客,特别是易感人群(老年人或患有慢性疾病的人)。应该把住所内有隔离人员的事实告诉给任何访客。
  \item 借助醒目的标记在住所内清晰地划出不同的区域。患者睡过的卧室或使用过的设施都标为红区。与红区相连的区域(比如客厅)则划为黄区。其他独立分开的房间是绿区。
  \item 患者应该严格遵守打喷嚏的礼仪,使用一次性纸巾遮掩口鼻,然后安全地丢弃,或者弯曲肘部(比如用袖子)遮掩口鼻,之后马上清洗衣物。
  \item 在住所中建立起沟通机制,以便患者在必须离开自己的房间前可以通知其他住客。
  \item 患者应定期对红区进行消毒。其他的同居者应定期对黄区(最好还对绿区)进行消毒。
  \item 患者应将自己限制在红区内活动,尽量减少进入黄区。另外需要完全避开绿区。对待患者密切接触的物体也应该遵循同样的原则。在红区外活动时,患者应佩戴手套和口罩。
  \item 可能的传播途径:共用设施:厨房、浴室等。共用家居用品:毛巾、眼镜、餐具等。共享的食物、饮料等。多人触摸的表面:门把手、桌子表面、遥控器、电灯开关等。这些地方应该每天至少消毒一次。
  \item 患者使用任何共用设施后都应该进行清洁,尤其是在使用完洗手间后。请在不使用抽水马桶时盖上盖子。
  \item 患者应该另外单独使用一个垃圾袋或垃圾箱,用于处理一次性手套、口罩、纸巾等。
  \item 如有可能,同居者应该在患者居住房间的门口帮忙取送快递,最大程度地减少患者离开红区的需求。

\end{itemize}





\end{multicols}



% \bibliography{MyCollection.bib}


\end{document}
