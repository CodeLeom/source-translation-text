\documentclass{article}

\usepackage[utf8]{inputenc}

\title{Leitlinie zur Selbstisolation}
\author{Chen Shen und Yaneer Bar-Yam \\ New England Complex Systems Institute \\ (übersetzt von K. Weisser, V. Brunsch)}

\date{März 2020}



\begin{document}

\maketitle


Diese Leitlinie ist für Individuen, die positiv auf COVID-19 getestet wurden, aber milde oder keine Symptome haben. Milde Symptome sind niedriges Fieber, leichte Abgeschlagenheit, Husten, aber ohne Symptome einer Lungenentzündung und ohne begleitende, chronische Erkrankungen. Diese Leitlinie kann auch eine Hilfestellung sein, wenn medizinische Ressourcen örtlich stark belastet sind und Individuen mit mittleren oder stärkeren Symptomen nicht in Krankenhäusern versorgt werden können. Es gilt zu beachten, dass diese Leitlinie für Individuen ohne oder mit nur milden Symptomen entworfen wurde. Sobald sich Symptome wie Atembeschwerden oder hohes Fieber einstellen, begeben Sie sich sofort in medizinische Versorgung.


\begin{center}GENERELL\end{center}
\begin{itemize}
\item Die meisten Fälle heilen mit einer vollständigen Gesundung aus. Quarantäne ist eine vorübergehende Maßnahme, die typischerweise 14 Tage andauert.
\item \textbf{Es wird dringend empfohlen, sich alleine zu isolieren oder zumindest in einem einzelnen Raum.}




\end{itemize}
\begin{center}WENN SIE ALLEINE LEBEN\end{center}
\begin{itemize}
\item Behalten Sie Ihren Gesundheitszustand im Auge. Führen Sie ein Gesundheitstagebuch mit deutlicher Handschrift und bewahren Sie es zentral auf. Das Tagebuch sollte enthalten: Datum und Uhrzeit des Eintrags, Puls, Sauerstoffsättigung (gemessen mit einem Oximeter), etwaige Symptome, Mahlzeiten, eingenommene Medikamente und deren Dosierung.
\item Halten Sie sich über lokale Informationen zum Vorgehen beim Auftreten von Symptomen auf dem Laufenden. Speichern Sie sich wichtige Kontaktdaten in die Schnellwahlliste ein. Richten Sie einen regelmäßigen Kontakt (mindestens täglich) mit einem Familienmitglied oder Freund ein. Teilen Sie diesen Personen Ihre Notfallkontakte mit und, wie Sie in Ihr Zuhause gelangen können, falls sie hilflos sind.
\item Fördern Sie Ihren Gesundheitszustand, indem Sie sich hydriert halten, sich ausgewogen ernähren und regelmäßig schlafen. Behalten Sie unterhaltsame oder weiterbildende Aktivitäten wie Lesen, Spiele oder den Austausch online bei oder entwickeln Sie diese.
\item Waschen Sie sich regelmäßig die Hände mit Seife oder Desinfektionsmittel für mindestens 20 Sekunden.
\item Es ist wichtig, Wohnräume regelmäßig zu lüften.
\item Waschen Sie Bettwäsche, Handtücher und Kleidung regelmäßig. Waschen Sie Ihre Wäsche getrennt von anderer Wäsche.
\item Vermeiden Sie Kontakt zu Ihren Haustieren und anderen Tieren. Wenn das nicht möglich ist, tragen Sie eine Maske und waschen Sie sich vor und nach dem Kontakt mit Tieren die Hände.
\item Isolieren Sie sich von physischem Kontakt mit anderen Menschen, aber bleiben Sie in Kontakt mit Familie und Freunden über SMS, das Telefon, Videoanrufe oder andere elektronische Kommunikationsmittel. Dies ist aus vielen Gründen wichtig, unter anderem dafür, einen positiven Ausblick zu haben.
\item Wenn Nachrichten über die Entwicklung der Pandemie Ihnen Sorge bereiten, versuchen Sie, sich nicht zu stark darauf zu konzentrieren, um die mentale Belastung gering zu halten.
\item Schaffen Sie sich eine relativ beständige, tägliche Routine. Wenn möglich, behalten Sie Sport auf moderatem oder auch geringem Level bei.
\item Koordinieren Sie mit Familie oder Freunden oder den lokalen Behörden die Lieferung des Grundbedarfs inklusive der täglichen Mahlzeiten. Behalten Sie Ihre Vorräte im Auge und informieren Sie rechtzeitig, wenn diese zur Neige gehen. Eine kontaktlose Übergabe ist zu bevorzugen. Eine Maske und Handschuhe sind notwendig, wenn Sie mit Lieferpersonal interagieren.
\item Besprechen Sie sich mit lokalen Behörden und medizinischem Fachpersonal hinsichtlich der Dauer und der Umstände einer Beendigung der Selbstisolation. Führen Sie vor der Beendigung eine Grundreinigung durch und waschen Sie alles. Lassen Sie auch nach Beendigung der Selbstisolation Vorsicht walten.
\item In Vorstädten oder ländlichen Gegenden, wo das Betreten und Verlassen Ihrer Behausung ohne Kontakt mit anderen Menschen oder mit geteilten Räumen möglich ist, sind Spaziergänge alleine möglich. Vergessen Sie dabei nicht, dass Sie in Quarantäne sind: Verzichten Sie auf Kontakt mit anderen Menschen und begeben Sie sich nicht an Orte, wo andere sich gerade oder später aufhalten könnten.
\end{itemize}
  
  
  
\begin{center}WENN SIE MIT ANDEREN ZUSAMMEN LEBEN\end{center}
\begin{itemize}
\item Die Menschen, mit denen Sie in häuslicher Gemeinschaft leben, gelten als "direkte Kontake" und sollten sich an die örtlichen Richtlinien bezüglich solcher direkter Kontakte halten. Dazu gehört vor allem, den Kontakt mit Außenstehenden zu vermeiden, wenn möglich.
\item Die anderen Mitglieder Ihrer häuslichen Gemeinschaft sollten Besuch vermeiden, besonders von besonders gefährdeten Menschen (ältere Menschen, oder solche mit chronischen Erkrankungen). Jeder Besucher sollte über die Anwesenheit einer Person in Quarantäne informiert sein.
\item Markieren Sie deutlich, wenn möglich mit visueller Hilfe, verschiedene Zonen in Ihrem Zuhause. Das Schlafzimmer und Räumlichkeiten, die vom Patienten genutzt werden, gelten als "Rote Zone". Mit der Roten Zone verbundene Räumlichkeiten, zum Beispiel das Wohnzimmer, gelten als "Gelbe Zone". Andere, separate Räumlichkeiten gelten als "Grüne Zone".
\item Der Patient sollte strikt die Niesetikette einhalten und zum Schneuzen Einmaltaschentücher benutzen oder Stoffe (z.B. den Ärmel), die bald gewaschen werden.
\item Schaffen Sie einen Kommunikationsmechanismus im Haushalt, durch den der Patient andere Bewohner informieren kann, wenn er/sie seinen/ihren Raum verlassen muss.
\item Der Patient sollte Rote Zonen regelmäßig desinfizieren. Andere Bewohner desinfizieren die Gelbe Zone und, wenn möglich, auch die Grüne Zone regelmäßig.
\item Der Patient sollte sich auf die Rote Zone beschränken und das Betreten der Gelben Zone minimieren. Die Grüne Zone ist gänzlich zu meiden. Gegenstände, die in Kontakt mit dem Patienten kommen, sollten ebenso behandelt werden. Außerhalb der Roten Zone sollte der Patient Maske und Handschuhe tragen.
\item Mögliche Übertragungswege - Geteilte Räumlichkeiten: Küche, Badezimmer, etc. Geteilte Haushaltswaren: Handtücher, Gläser, Untensilien, etc. Geteilte Nahrungsmittel oder Getränke, etc. Berührte Oberflächen: Türgriffe, Tischoberflächen, Fernbedienung, Lichtschalter, etc. Diese Dinge sollten mindestens ein Mal am Tag desinfiziert werden.
\item Der Patient sollte alle geteilten Räumlichkeiten nach Benutzung reinigen. Die Toilette sollte geschlossen bleiben, wenn sie nicht genutzt wird.
\item Der Patient sollte einen separaten Abfall haben, um Handschuhe, Masken, Taschentücher, etc. zu entsorgen.
\item Wenn möglich, sollten Mitbewohner dem Patienten bestelltes Essen oder Pakete an die Tür bringen, um den Bedarf des Patienten, die Rote Zone zu verlassen, zu minimieren.
\end{itemize}

\end{document}
