%\documentclass[twocolumn,journal]{IEEEtran}
\documentclass[onecolumn,journal]{IEEEtran}
\usepackage{amsfonts}
\usepackage{amsmath}
\usepackage{amsthm}
\usepackage{amssymb}
\usepackage{graphicx}
\usepackage[T1]{fontenc}
%\usepackage[english]{babel}
\usepackage{supertabular}
\usepackage{longtable}
\usepackage[usenames,dvipsnames]{color}
\usepackage{bbm}
%\usepackage{caption}
\usepackage{fancyhdr}
\usepackage{breqn}
\usepackage{fixltx2e}
\usepackage{capt-of}
%\usepackage{mdframed}
\setcounter{MaxMatrixCols}{10}
\usepackage{tikz}
\usetikzlibrary{matrix}
\usepackage{endnotes}
\usepackage{soul}
\usepackage{marginnote}
%\newtheorem{theorem}{Theorem}
\newtheorem{lemma}{Lemma}
%\newtheorem{remark}{Remark}
%\newtheorem{error}{\color{Red} Error}
\newtheorem{corollary}{Corollary}
\newtheorem{proposition}{Proposition}
\newtheorem{definition}{Definition}
\newcommand{\mathsym}[1]{}
\newcommand{\unicode}[1]{}
\newcommand{\dsum} {\displaystyle\sum}
\hyphenation{op-tical net-works semi-conduc-tor}
\usepackage{pdfpages}
\usepackage{enumitem}
\usepackage{multicol}
\usepackage[utf8]{inputenc}

\headsep = 5pt
\textheight = 730pt
%\headsep = 8pt %25pt
%\textheight = 720pt %674pt
%\usepackage{geometry}

\bibliographystyle{unsrt}

\usepackage{float}

 \usepackage{xcolor}

\usepackage[framemethod=TikZ]{mdframed}
%%%%%%%FRAME%%%%%%%%%%%
\usepackage[framemethod=TikZ]{mdframed}
\usepackage{framed}
    % \BeforeBeginEnvironment{mdframed}{\begin{minipage}{\linewidth}}
     %\AfterEndEnvironment{mdframed}{\end{minipage}\par}


%	%\mdfsetup{%
%	%skipabove=20pt,
%	nobreak=true,
%	   middlelinecolor=black,
%	   middlelinewidth=1pt,
%	   backgroundcolor=purple!10,
%	   roundcorner=1pt}

\mdfsetup{%
	outerlinewidth=1,skipabove=20pt,backgroundcolor=yellow!50, outerlinecolor=black,innertopmargin=0pt,splittopskip=\topskip,skipbelow=\baselineskip, skipabove=\baselineskip,ntheorem,roundcorner=5pt}

\mdtheorem[nobreak=true,outerlinewidth=1,%leftmargin=40,rightmargin=40,
backgroundcolor=yellow!50, outerlinecolor=black,innertopmargin=0pt,splittopskip=\topskip,skipbelow=\baselineskip, skipabove=\baselineskip,ntheorem,roundcorner=5pt,font=\itshape]{result}{Result}


\mdtheorem[nobreak=true,outerlinewidth=1,%leftmargin=40,rightmargin=40,
backgroundcolor=yellow!50, outerlinecolor=black,innertopmargin=0pt,splittopskip=\topskip,skipbelow=\baselineskip, skipabove=\baselineskip,ntheorem,roundcorner=5pt,font=\itshape]{theorem}{Theorem}

\mdtheorem[nobreak=true,outerlinewidth=1,%leftmargin=40,rightmargin=40,
backgroundcolor=gray!10, outerlinecolor=black,innertopmargin=0pt,splittopskip=\topskip,skipbelow=\baselineskip, skipabove=\baselineskip,ntheorem,roundcorner=5pt,font=\itshape]{remark}{Remark}

\mdtheorem[nobreak=true,outerlinewidth=1,%leftmargin=40,rightmargin=40,
backgroundcolor=pink!30, outerlinecolor=black,innertopmargin=0pt,splittopskip=\topskip,skipbelow=\baselineskip, skipabove=\baselineskip,ntheorem,roundcorner=5pt,font=\itshape]{quaestio}{Quaestio}

\mdtheorem[nobreak=true,outerlinewidth=1,%leftmargin=40,rightmargin=40,
backgroundcolor=yellow!50, outerlinecolor=black,innertopmargin=5pt,splittopskip=\topskip,skipbelow=\baselineskip, skipabove=\baselineskip,ntheorem,roundcorner=5pt,font=\itshape]{background}{Background}

%TRYING TO INCLUDE Ppls IN TOC
\usepackage{hyperref}


\begin{document}
\title{\color{Brown}  Linee guida per le Aziende
\vspace{-0.35ex}}
\author{Chen Shen and Yaneer Bar-Yam \\ New England Complex Systems Institute \\
\vspace{+0.35ex}
\small{\textit{(translated by A. P. Rossi, P. Bonavita, E. Lamberti})}\\
 \today
  \vspace{-8ex} \\
%\bigskip
\textbf{}
 }

\maketitle

%\vspace{-1ex}
%\flushbottom % Makes all text pages the same height

%\maketitle % Print the title and abstract box

%\tableofcontents % Print the contents section

\thispagestyle{empty} % Removes page numbering from the first page

%----------------------------------------------------------------------------------------
%	ARTICLE CONTENTS
%----------------------------------------------------------------------------------------

%\section*{Introduction} % The \section*{} command stops section numbering

%\addcontentsline{toc}{section}{\hspace*{-\tocsep}Introduction} % Adds this section to the table of contents with negative horizontal space equal to the indent for the numbered sections

%\tableofcontents
%\section{ Introduction}

%\section*{Overview}


\begin{multicols}{2}

% \section

Elenchiamo di seguito una lista di azioni suggerite per le aziende al fine di prevenire la diffusione del Coronavirus. Sono fornite speciali raccomandazioni per il commercio al dettaglio e per il settore dell’ospitalità.


\section*{Aspetti generali}
\begin{itemize}
\item Promuovere la comprensione tra i dipendenti e le loro famiglie dei meccanismi di trasmissione del Coronavirus e la sua prevenzione
\item Sviluppare regole specifiche per ridurre la trasmissione, applicandole meticolosamente
\item Assicurarsi che i dipendenti sappiano che quando hanno sintomi anche molto lievi non devono recarsi a lavoro o partecipare fisicamente alle riunioni, e che non saranno penalizzati per i giorni di malattia. Creare un sistema per registrare ogni caso
\item Assicurarsi che i dipendenti abbiano appropriate regole di copertura sanitaria così da non esitare a chiedere aiuto medico in presenza di sintomi, anche molto lievi
\item Interagire con le strutture mediche locali per coordinare spediti controlli dei dipendenti per infezione da Coronavirus
\item Preparare strumenti essenziali (disinfettante mani, alcohol, mascherine\footnote{L’uso di mascherine e’ dibattuto. Facciamo presente che (1) ogni individuo che ha sintomi anche molto lievi deve evitare contatti con altre persone e deve indossare una mascherina in ogni attività che implica contatto con altre persone. (2) Indossare una mascherina deve essere accettato in luoghi pubblici per evitare a chi e’ malato di esitare o di sentirsi stigmatizzato/a per l’indossare una mascherina (3) Nonostante le mascherine non garantiscano sicurezza per un individuo sano e la loro disponibilità sia limitata a causa dei bisogni prioritari del settore medico, l’utilizzo di mascherine nei casi in cui la prossimità con altri individui potenzialmente non può essere evitata, riduce drammaticamente il rischio di infezione. (4) Per coloro che hanno più di 50 anni o soffrono di condizioni cliniche preesistenti, così come per coloro che si trovano in zone ad alto rischio, il costo di un eventuale contagio giustifica pienamente l’uso della mascherina.  }, termometri all’infrarosso senza contatto) in caso che la situazione peggiori rapidamente e gli impiegati non abbiano accesso a tali strumenti
\item Rafforzare i punti più deboli dell'organizzazione, riducendone la vulnerabilità complessiva
\end{itemize}

\section*{Riunioni, viaggi e visitatori}
\begin{itemize}
\item Sostituire le riunioni fisiche con mezzi di comunicazione virtuali
\item Permettere ai dipendenti di lavorare da casa quando possibile
\item Limitare i viaggi in aree ad alto rischio (Rosso, Arancio, anche Giallo)
\item Eliminare viaggi non essenziali
\item Modificare le modalità di lavoro per rendere viaggi apparentemente essenziali non più necessari
\item Limitare i visitatori in generale, e applicare regole per azzerare quelli che hanno relazioni di residenza e/o lavoro con le zone di focolaio. In ogni caso, controllare i sintomi all'arrivo.
\end{itemize}

\section*{Luoghi di lavoro}
\begin{itemize}
\item Promuovere ore di lavoro flessibili, turni per diminuire la densita’ di persone all’interno dei luoghi di lavoro. La densita’ va ridotta a meno del 50% della capacita’ in ogni momento
\item Le aziende devono chiedere ai dipendenti che tornano da zone con casi confermati, o che hanno avuto contatti sospettabili di infezione durante i viaggi di lavoro, di mettersi in quarantena volontaria per 14 giorni prima di tornare in ufficio. Le aziende devono controllare le condizioni di salute, e comunicare e cercare supporto medico se necessario
\item I punti di accesso devono essere dotati di guardie con termometri all’infrarosso senza contatto
\item Misurare la temperatura corporea dei dipendenti giornalmente e fornire loro mascherine quando la vicinanza con altri dipendenti non puo’ essere evitata
\item Modificare il traffico verso gli edifici per promuovere il lavaggio delle mani all’ingresso e piazzare disinfettante all’entrata di ogni ufficio
\item Organizzarsi affinché i dipendenti evitino di raggrupparsi negli ascensori. Gli ascensori non devono contenere piu’ della meta’ della propria capacita’massima
\item Assicurarsi che la postazione di lavoro di ogni dipendente sia separata di almeno 1 metro, e che ogni postazione individuale sia di almeno 2.5 m2. Per uffici con molti dipendenti, questi valori vanno aumentati
\item Disinfettare spazi comuni, locali trafficati, superfici toccate di frequente
\item Se viene usata l’aria condizionata, disabilitare il ricircolo. Pulire e disinfettare settimanalmente e cambiare i filtri
\item Distribuire i dipendenti ai pasti, tenendo distanza di 1 m durante i pasti ed evitando di sedersi uno di fronte all’altro. Separare gli utensili e disinfettare frequentemente. La salute del personale dei bar deve essere controllata frequentemente
\item Promuovere la consegna pasti rispetto al mangiare fuori. Fornire aiuto per ricevere pasti senza contatto, in una zona pulita e senza il formarsi di code.
\item Considerare come i dipendenti si recano a lavoro e sviluppare raccomandazioni, ad esempio: evitare il trasporto pubblico, curare l’igiene personale, evitare di toccare le superfici in luoghi pubblici, lavarsi le mani ed indossare mascherine nelle aree ad rischio elevato
\item Gli incarichi per il rispetto delle regole relative alla sicurezza da Coronavirus sul luogo di lavoro devono essere chiari e avere una chiara catena di responsabilità.
\end{itemize}
%
\section*{Punti vendita ed ospitalità}
\begin{itemize}
\item Le attivita’ commerciali caratterizzate da alti contatti possono essere gravemente colpite. Interventi celeri ed effettivi possono mitigare ma non eliminare il rischio, a meno che non coinvolgano l’intera societa’
\item È assolutamente fondamentale assicurarsi che coloro che presentano sintomi di raffreddamento anche lievi non lavorino a contatto con altre persone.
\item Mantenere un registro rigoroso dei contatti giornalieri, così che in caso sia scoperta un'infezione, l'azienda possa tempestivamente contattare tutte le persone potenzialmente esposte e minimizzare il rischio per clienti e dipendenti.
\item Metodi di lavoro che non richiedano contatto vanno sviluppati e implementati, inclusi:
  \begin{itemize}
    \item Consegnare/ricevere materiali all’ingresso/finestra, assicurandosi ci sia spazio abbastanza tra chi fa la fila.
    \item Raggiungere il servizio in auto
    \item Consegna a domicilio senza contatto
  \end{itemize}
\end{itemize}


\end{multicols}

\vspace{2ex}







% \bibliography{MyCollection.bib}
\bibliography{references.bib}

\end{document}
