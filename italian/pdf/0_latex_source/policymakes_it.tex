%\documentclass[twocolumn,journal]{IEEEtran}
\documentclass[onecolumn,journal]{IEEEtran}
\usepackage{amsfonts}
\usepackage{amsmath}
\usepackage{amsthm}
\usepackage{amssymb}
\usepackage{graphicx}
\usepackage[T1]{fontenc}
%\usepackage[english]{babel}
\usepackage{supertabular}
\usepackage{longtable}
\usepackage[usenames,dvipsnames]{color}
\usepackage{bbm}
%\usepackage{caption}
\usepackage{fancyhdr}
\usepackage{breqn}
\usepackage{fixltx2e}
\usepackage{capt-of}
%\usepackage{mdframed}
\setcounter{MaxMatrixCols}{10}
\usepackage{tikz}
\usetikzlibrary{matrix}
\usepackage{endnotes}
\usepackage{soul}
\usepackage{marginnote}
%\newtheorem{theorem}{Theorem}
\newtheorem{lemma}{Lemma}
%\newtheorem{remark}{Remark}
%\newtheorem{error}{\color{Red} Error}
\newtheorem{corollary}{Corollary}
\newtheorem{proposition}{Proposition}
\newtheorem{definition}{Definition}
\newcommand{\mathsym}[1]{}
\newcommand{\unicode}[1]{}
\newcommand{\dsum} {\displaystyle\sum}
\hyphenation{op-tical net-works semi-conduc-tor}
\usepackage{pdfpages}
\usepackage{enumitem}
\usepackage{multicol}
\usepackage[utf8]{inputenc}


\headsep = 5pt
\textheight = 730pt
%\headsep = 8pt %25pt
%\textheight = 720pt %674pt
%\usepackage{geometry}

\bibliographystyle{unsrt}

\usepackage{float}

 \usepackage{xcolor}

\usepackage[framemethod=TikZ]{mdframed}
%%%%%%%FRAME%%%%%%%%%%%
\usepackage[framemethod=TikZ]{mdframed}
\usepackage{framed}
    % \BeforeBeginEnvironment{mdframed}{\begin{minipage}{\linewidth}}
     %\AfterEndEnvironment{mdframed}{\end{minipage}\par}


%	%\mdfsetup{%
%	%skipabove=20pt,
%	nobreak=true,
%	   middlelinecolor=black,
%	   middlelinewidth=1pt,
%	   backgroundcolor=purple!10,
%	   roundcorner=1pt}

\mdfsetup{%
	outerlinewidth=1,skipabove=20pt,backgroundcolor=yellow!50, outerlinecolor=black,innertopmargin=0pt,splittopskip=\topskip,skipbelow=\baselineskip, skipabove=\baselineskip,ntheorem,roundcorner=5pt}

\mdtheorem[nobreak=true,outerlinewidth=1,%leftmargin=40,rightmargin=40,
backgroundcolor=yellow!50, outerlinecolor=black,innertopmargin=0pt,splittopskip=\topskip,skipbelow=\baselineskip, skipabove=\baselineskip,ntheorem,roundcorner=5pt,font=\itshape]{result}{Result}


\mdtheorem[nobreak=true,outerlinewidth=1,%leftmargin=40,rightmargin=40,
backgroundcolor=yellow!50, outerlinecolor=black,innertopmargin=0pt,splittopskip=\topskip,skipbelow=\baselineskip, skipabove=\baselineskip,ntheorem,roundcorner=5pt,font=\itshape]{theorem}{Theorem}

\mdtheorem[nobreak=true,outerlinewidth=1,%leftmargin=40,rightmargin=40,
backgroundcolor=gray!10, outerlinecolor=black,innertopmargin=0pt,splittopskip=\topskip,skipbelow=\baselineskip, skipabove=\baselineskip,ntheorem,roundcorner=5pt,font=\itshape]{remark}{Remark}

\mdtheorem[nobreak=true,outerlinewidth=1,%leftmargin=40,rightmargin=40,
backgroundcolor=pink!30, outerlinecolor=black,innertopmargin=0pt,splittopskip=\topskip,skipbelow=\baselineskip, skipabove=\baselineskip,ntheorem,roundcorner=5pt,font=\itshape]{quaestio}{Quaestio}

\mdtheorem[nobreak=true,outerlinewidth=1,%leftmargin=40,rightmargin=40,
backgroundcolor=yellow!50, outerlinecolor=black,innertopmargin=5pt,splittopskip=\topskip,skipbelow=\baselineskip, skipabove=\baselineskip,ntheorem,roundcorner=5pt,font=\itshape]{background}{Background}

%TRYING TO INCLUDE Ppls IN TOC
\usepackage{hyperref}


\begin{document}
\title{\color{Brown} COVID-19: Raccomandazioni per i Legislatori
 \\
\vspace{-0.35ex}}
\author{Chen Shen e Yaneer Bar-Yam \\ New England Complex Systems Institute \\
\vspace{+0.35ex}
\small{\textit{(tradotto da S. Vitale; A. P. Rossi})}\\
 \today
  \vspace{-14ex} \\


\bigskip
\bigskip

\textbf{}
 }

\maketitle


\flushbottom % Makes all text pages the same height

%\maketitle % Print the title and abstract box

%\tableofcontents % Print the contents section

\thispagestyle{empty} % Removes page numbering from the first page

%----------------------------------------------------------------------------------------
%	ARTICLE CONTENTS
%----------------------------------------------------------------------------------------

%\section*{Introduction} % The \section*{} command stops section numbering

%\addcontentsline{toc}{section}{\hspace*{-\tocsep}Introduction} % Adds this section to the table of contents with negative horizontal space equal to the indent for the numbered sections

%\tableofcontents
%\section{ Introduction}
\renewcommand{\thefootnote}{\fnsymbol{footnote}}




\begin{multicols}{2}

  \section*{La sfida}

  Il COVID-19 è una malattia a trasmissione rapida, che richiede l’ospedalizzazione in circa il 20\% dei casi, la terapia intensiva nel 10\% e ha un decorso fatale nel 2-4\% dei casi.
  Le complicazioni insorgono rapidamente per gli ultracinquantenni con stati di salute già compromessi, ad esempio da malattie cardiache o vascolari, ulteriormente aumentando il rischio.

  Il COVID-19 si può trasmettere anche avendo sintomi lievi (tosse, starnuti o febbre alta) e, verosimilmente, anche prima che i sintomi si manifestino.

  L’epidemia di COVID-19 conta, attualmente, più casi di quanti siano stati scoperti (è la punta dell’iceberg) e gli stessi stanno crescendo rapidamente:

\begin{itemize}

\item In assenza di un intervento sufficientemente efficace, il moltiplicatore
giornaliero è del 1.5x (Cina: 21 gennaio – 27 gennaio; Corea del Sud:
19 febbraio – 22 febbraio; Iran: 22 febbraio – 3 marzo; Danimarca:
26 febbraio – 9 marzo). Pertanto, avendo 100 nuovi casi oggi,
ve ne saranno 1,700 in una settimana e 29,000 in due settimane.

\item Se si agisce in modo da ridurre il moltiplicatore fino a 1,1x, allora avendo 100 nuovi casi oggi, il numero di nuovi casi in una settimana sarà 195, in due settimane 380.

\item  Riducendo il moltiplicatore a 1, ci saranno 100 nuovi casi in una settimana, 100 in due settimane.

\item  Infine, attivandosi per abbassare il moltiplicatore allo 0,9x, avendo 100 casi oggi, il numero di nuovi casi in una settimana sarà di 48, in due settimane 23 e saremmo sulla buona strada per sconfiggere l’epidemia.

\end{itemize}

  La crescita rapida significa che il numero di casi sembra irrilevante fino al momento in cui, all’improvviso, essa supera la capacità di risposta sanitaria. Questo include il numero di letti in ospedale e financo la capacità di mantenere i normali servizi di base.

  A causa del ritardo tra la trasmissione e il manifestarsi dei sintomi, tutti gli effetti della prevenzione sono posticipati di circa 4 giorni. Anche se in questo momento i cittadini fossero del tutto isolati in ambienti sterili, l’aumento giornaliero procederebbe per inerzia per circa altri 4 giorni.

  Le persone sono connesse da un’invisibile rete di trasmissione, i cui legami sono: i contatti fisici tra gli individui; il respirare la stessa aria che contiene corpuscoli espulsi con tosse, starnuti o semplicemente con l’espirazione; e persino oggetti che possono trasportare particelle virali depositate sugli stessi e successivamente toccate da altri. Questa rete di trasmissione opera continuamente durante le nostre attività quotidiane. Essa include contatti sul lavoro e in famiglia, tra amici e nella propria comunità.

  Il modo in cui la rete è connessa tra gli individui, determina il rischio che un individuo possa contrarre la malattia e trasmetterla ad altri.

  La chiave per ridurre il moltiplicatore è di tagliare radicalmente la rete di trasmissione.

\section*{INTERVENTI RACCOMANDATI}

Il nostro appello è rivolto a tutti i legislatori e i rappresentati dei cittadini affinché siano implementate, senza alcun indugio, le seguenti azioni:

\begin{enumerate}

\item Limitare il trasporto tra nazione e nazione, e tra diverse parti della stessa nazione, richiedendo almeno 14 giorni di quarantena per coloro che si spostano da regione a regione. Una strategia di isolamento e contenimento è essenziale.

\item Lavorare in collaborazione con istituti sanitari, aziende private e istituzioni accademiche per aumentare in modo rapido e radicale il numero di test, al fine di individuare gli individui da porre in isolamento. Vi sono diversi laboratori nelle università e in società private che possono effettuare test e salvare delle vite.

\item Isolare comunità con focolai di trasmissione attivi - ad oggi interi Paesi in Europa. In queste zone, chiunque, ad eccezione di coloro che forniscono servizi pubblici essenziali deve restare in casa. Svolgere servizi porta a porta (con adeguati dispositivi di protezione personale - EPP) per verificare la presenza di nuovi casi e la necessità di servizi, con il coinvolgimento della comunità.

\item Incoraggiare le imprese a mantenere in essere solo le funzioni essenziali e ridurre l’impatto su tutte le funzioni aziendali utilizzando postazioni di lavoro opportunamente distanziate, massimizzando lo smart-working da casa al fine di consentire un auto-isolamento e promuovere la creazione di spazi sicuri per individui e famiglie.

\item Aumentare la capacità di risposta sanitaria convertendo temporaneamente spazi pubblici e privati in ospedali per curare casi con sintomi lievi e moderati, al fine di agevolare la separazione dei soggetti infetti dal resto della popolazione. Incrementare le unità di terapia intensiva nel più breve tempo possibile.

\item Monitorare, proteggere e fronteggiare i bisogni delle fasce più deboli della popolazione, inclusi i senza-tetto, nonché i luoghi ad alta densità abitativa (incluso prigioni, case di cura, ospizi, dormitori e ospedali psichiatrici).

\item  Valutare attentamente l’entità delle risorse e delle scorte mediche, al fine di stimare possibili carenze dettate dalla crescita esponenziale delle necessità. Avviare immediatamente un’azione di mitigazione di tali carenze. Accumulare scorte di risorse essenziali, riconvertire attività produttive per colmare eventuali carenze, favorendo la protezione del personale medico.

\item {Collaborare attivamente con la comunità internazionale su nuovi metodi di intervento (come, ad esempio, i drive-through test sperimentati in Corea). Ci troviamo di fronte ad una situazione nuova e fluida e le innovazioni sono testate e implementate continuamente a livello globale.}

\item Allentare l’applicazione di certe regole e regolamenti utilizzati in situazioni di “normalità”, che non si attagliano all’attuale condizione. Privilegiare la rapidità di risposta e la proattività alla ricerca della soluzione perfetta. È essenziale una comunicazione attenta e trasparente che promuova il coinvolgimento della popolazione e la partecipazione attiva della stessa nella propria sicurezza.


\end{enumerate}



%
% \section*{II. Auto-isolamento}
%
% \section*{III. Famiglia e amici}
%
% \section*{IV. Comunità}
%
% \section*{V. Aziende}
%
% \section*{VI. Strutture sanitarie}
%
% \section*{VI. Amministrazioni, Governi, Commercio}



%
% In aree a rischio elevato laddove il governo non prende misure adeguate, proteggere la famiglia, o la collettività e’ difficile. La diffusione di un incendio Affinché un incendio si diffonda, è necessario che gli elementi combustibili vengano in qualche modo in contatto tra loro. Analogamente, il contagio di COVID-19 necessita di una catena di individui suscettibili. La soluzione e’: (1) ridurre i contatti tra la famiglia e gli altri individui, e provvedere ai bisogni essenziali, quando il rischio aumenta. (2) creare uno spazio sicuro che protegge chi si trova al suo interno, con un accordo condiviso di non essere in contatto fisico non protetto con altri e con superfici che sono toccate da altri.
%
%
% Lo spazio sicuro contribuisce a ridurre il contagio anche perché chi e’ al suo interno non partecipa alla trasmissione della malattia. I membri di uno spazio sicuro si possono unire ai membri di altri spazi sicuri per espandere gradualmente lo spazio e crearne di nuovi. Seguono le linee guida per le famiglie.
%
% Ridurre il contatto tra la famiglia e gli altri:
%
% \begin{itemize}
%
% \item Leggere accuratamente le nostre linee guida per gli individui e condividerle con i familiari. Discutere con loro come ridurre i loro contatti con gli altri
%
% \item Rendere le riunioni familiari virtuali. La presente epidemia o sarà sconfitta, o diventerà diffusa. Nel primo caso, in pochi mesi torneremo alla normalità Nel caso di diffusione, azioni differenti saranno necessarie.
%
% \item Assicuratevi che voi ed i membri della vostra famiglia disponiate dei necessari beni di consumo, inclusa una scorta di medicine. Considerate i membri vulnerabili della famiglia, inclusi gli anziani, ma anche chiunque sopra i 50 anni di eta’, e coloro che soffrono di malattie croniche, come a rischio per il contatto con altri. Cercate di ridurre i loro contatti, e di supportarli in modo tale da farli restare a casa il più possibile e non dover recarsi in spazi affollati.
%
% \item Valutare il trasferimento temporaneo di persone da abitazioni condivise (case di riposo, strutture abitative assistite, case di degenza, etc.) a sistemazioni più’ isolate, ad esempio case private, o strutture più piccole
%
% \item Laddove non fosse possibile ridurre i contatti tra gli ospiti, discutere con i responsabili delle strutture la necessità di incrementare i livelli di precauzione contro il virus.
% \item Evitare assembramenti e posti affollati, tra cui eventi pubblici e ristoranti specialmente se in spazi chiusi
%
% \end{itemize}
%
% Creare Spazi Sicuri in situazioni ad elevato rischio:
%
% \begin{itemize}
%
% \item L’obiettivo principale di stabilire uno spazio sicuro tra un gruppo di persone e’ quello di creare un'unità’ isolata che riduca i contatti fisici con altre individui esterni all'unità’ al minimo indispensabile, al tempo stesso essendo in grado di auto-sostenersi e auto-supportarsi.
%
% \item Non c’e’ motivo per cui i singoli individui debbano aspettare la decisione di misure di sicurezza ‘top-down’ promulgate dai governi. In assenza di interventi drastici e sistematici, gli spazi sicuri autogestiti e organizzati ‘dal basso’ possono essere d’aiuto. Con il crescere del loro numero, le zone sicure possono rallentare o addirittura spegnere i focolai d’infezione locali.
%
% \item Gli Spazi Sicuri possono essere costituiti da famiglie o gruppi che decidano di condividere una struttura abitativa. Più strutture abitative possono essere combinate, ad esempio spostandosi tra l’una e l’altra (a piedi o in macchina), se sono stabiliti e seguiti stringenti protocolli di sicurezza. Per avviare con successo uno Spazio Sicuro, OGNI partecipante deve accettare e rispettare la promessa di minimizzare i contatti fisici esterni al gruppo. Devono essere inoltre stabilite istruzioni chiare su come comportarsi e cooperare. I membri di uno Spazio Sicuro devono essere aperti e trasparenti sulle proprie condizioni di salute e la propria cronologia di viaggio, ed essere reciprocamente responsabile della salute degli altri membri.
%
% \item Per far impegnare gli individui a condividere uno spazio condiviso, alcuni aspetti vanno organizzati con le aziende, le scuole, la famiglia e gli amici. Potrebbe rendersi necessario, stare a casa con l'approvazione dell'azienda, o assentarsi dal lavoro.
%
% \item Pianificare un periodo esteso (almeno una o più’ settimane) in uno spazio sicuro deve essere fatto in anticipo, incluso il reperimento di beni di consumo, ma durante l'approvvigionamento bisogna porre particolare attenzione, data la potenziale esposizione al un gran numero di persone. Formulare delle strategie di sopravvivenza può essere utili in questo ambito. Avere un piano relativo alla gestione dei bisogni è vitale perché la ricerca di approvvigionamenti comporta necessariamente del rischio.
%
% \item Quando possibile, organizzare la consegna di beni, incluso il cibo, per limitare i trasferimenti verso i negozi. Bisogna fare attenzione perché’ i beni consegnati sono stati toccati da qualcuno. A meno che non ci sia un accordo con il fornitore riguardo all’uso di guanti, lavare e disinfettare i beni ricevuti e’ consigliato  nelle aree di trasmissione attiva.
%
% \item Per attività essenziali, inclusi gli acquisti, durante le quali un contatto fisico esterno è inevitabile, i membri della famiglia devono pianificare in anticipo per agire in modo efficiente e minimizzare la durata e l’estensione del contatto. Uscire e tornare nello spazio sicuro necessita di precauzioni. Usare protezione personale appropriata, inclusi guanti e materiale monouso (fazzoletti) per prendere o manipolare oggetti che non devono essere toccati, disinfettante o alcool per le mani, e mascherine. Ritornare allo spazio sicuro necessità di lavarsi e disinfettarsi, preferibilmente prima del rientro.
%
% \item Promuovere la comunicazione interna e la mutua attenzione è essenziale per mantenere i membri dello spazio sicuro in una relazione positiva ed in buona salute mentale. Riconoscere che l’emergenza corrente richiede azioni straordinarie e sacrifici e’ fondamentale. Questo può mitigare, anche se non sostituire completamente, l’importanza del supporto reciproco.
%
% \item I membri dello spazio sicuro devono ottenere informazioni sulle azioni da intraprendere in caso uno o più membri mostrino sintomi di infezione. Le azioni variano in base al paese/luogo, e devono essere peraltro dinamiche secondo la situazione. I membri devono informarsi reciprocamente all’interno del gruppo a riguardo del piano contingente più aggiornato e le informazioni sui contatti avuti. Nel caso che un membro mostri sintomi tipici, gli altri devono agire velocemente per aiutarlo/a a farsi testare, e praticare l’isolamento precauzionale nel periodo precedente all’ottenimento dei risultati.
%
% \end{itemize}
%
% Quando una epidemia progredisce, decisioni difficili dovranno inevitabilmente essere prese a riguardo dell’uscita da uno spazio sicuro per aiutare membri della famiglia o amici che non sono in uno spazio sicuro. Gli individui devono essere preparati a fare queste decisioni.
%
% In un periodo di rischio elevato, ci saranno inevitabilmente azioni prese per sbaglio che potrebbero compromettere la sicurezza. Per evitare reazioni eccessive a un evento singolo, è importante capire che ogni singolo atto ha una piccola probabilità di creare danni. Tuttavia, quando azioni multiple vengono prese, il rischio aumenta drammaticamente. Assicurarsi che le relative lezioni vengano apprese è più importante di accusare, incolpare o punire.


\end{multicols}



% \bibliography{MyCollection.bib}


\end{document}
