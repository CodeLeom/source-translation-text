%\documentclass[twocolumn,journal]{IEEEtran}
\documentclass[onecolumn,journal]{IEEEtran}
\usepackage{amsfonts}
\usepackage{amsmath}
\usepackage{amsthm}
\usepackage{amssymb}
\usepackage{graphicx}
\usepackage[T1]{fontenc}
%\usepackage[english]{babel}
\usepackage{supertabular}
\usepackage{longtable}
\usepackage[usenames,dvipsnames]{color}
\usepackage{bbm}
%\usepackage{caption}
\usepackage{fancyhdr}
\usepackage{breqn}
\usepackage{fixltx2e}
\usepackage{capt-of}
%\usepackage{mdframed}
\setcounter{MaxMatrixCols}{10}
\usepackage{tikz}
\usetikzlibrary{matrix}
\usepackage{endnotes}
\usepackage{soul}
\usepackage{marginnote}
%\newtheorem{theorem}{Theorem}
\newtheorem{lemma}{Lemma}
%\newtheorem{remark}{Remark}
%\newtheorem{error}{\color{Red} Error}
\newtheorem{corollary}{Corollary}
\newtheorem{proposition}{Proposition}
\newtheorem{definition}{Definition}
\newcommand{\mathsym}[1]{}
\newcommand{\unicode}[1]{}
\newcommand{\dsum} {\displaystyle\sum}
\hyphenation{op-tical net-works semi-conduc-tor}
\usepackage{pdfpages}
\usepackage{enumitem}
\usepackage{multicol}
\usepackage[utf8]{inputenc}

\headsep = 5pt
\textheight = 730pt
%\headsep = 8pt %25pt
%\textheight = 720pt %674pt
%\usepackage{geometry}

\bibliographystyle{unsrt}

\usepackage{float}

 \usepackage{xcolor}

\usepackage[framemethod=TikZ]{mdframed}
%%%%%%%FRAME%%%%%%%%%%%
\usepackage[framemethod=TikZ]{mdframed}
\usepackage{framed}
    % \BeforeBeginEnvironment{mdframed}{\begin{minipage}{\linewidth}}
     %\AfterEndEnvironment{mdframed}{\end{minipage}\par}


%	%\mdfsetup{%
%	%skipabove=20pt,
%	nobreak=true,
%	   middlelinecolor=black,
%	   middlelinewidth=1pt,
%	   backgroundcolor=purple!10,
%	   roundcorner=1pt}

\mdfsetup{%
	outerlinewidth=1,skipabove=20pt,backgroundcolor=yellow!50, outerlinecolor=black,innertopmargin=0pt,splittopskip=\topskip,skipbelow=\baselineskip, skipabove=\baselineskip,ntheorem,roundcorner=5pt}

\mdtheorem[nobreak=true,outerlinewidth=1,%leftmargin=40,rightmargin=40,
backgroundcolor=yellow!50, outerlinecolor=black,innertopmargin=0pt,splittopskip=\topskip,skipbelow=\baselineskip, skipabove=\baselineskip,ntheorem,roundcorner=5pt,font=\itshape]{result}{Result}


\mdtheorem[nobreak=true,outerlinewidth=1,%leftmargin=40,rightmargin=40,
backgroundcolor=yellow!50, outerlinecolor=black,innertopmargin=0pt,splittopskip=\topskip,skipbelow=\baselineskip, skipabove=\baselineskip,ntheorem,roundcorner=5pt,font=\itshape]{theorem}{Theorem}

\mdtheorem[nobreak=true,outerlinewidth=1,%leftmargin=40,rightmargin=40,
backgroundcolor=gray!10, outerlinecolor=black,innertopmargin=0pt,splittopskip=\topskip,skipbelow=\baselineskip, skipabove=\baselineskip,ntheorem,roundcorner=5pt,font=\itshape]{remark}{Remark}

\mdtheorem[nobreak=true,outerlinewidth=1,%leftmargin=40,rightmargin=40,
backgroundcolor=pink!30, outerlinecolor=black,innertopmargin=0pt,splittopskip=\topskip,skipbelow=\baselineskip, skipabove=\baselineskip,ntheorem,roundcorner=5pt,font=\itshape]{quaestio}{Quaestio}

\mdtheorem[nobreak=true,outerlinewidth=1,%leftmargin=40,rightmargin=40,
backgroundcolor=yellow!50, outerlinecolor=black,innertopmargin=5pt,splittopskip=\topskip,skipbelow=\baselineskip, skipabove=\baselineskip,ntheorem,roundcorner=5pt,font=\itshape]{background}{Background}

%TRYING TO INCLUDE Ppls IN TOC
\usepackage{hyperref}


\begin{document}
\title{\color{Brown}  Diretrizes para Empresas
\vspace{-0.35ex}}
\author{Chen Shen e Yaneer Bar-Yam \\ New England Complex Systems Institute \\
\vspace{+0.35ex}
\small{\textit{(traduzido por Lucas Pontes})}\\
 \today
  \vspace{-8ex} \\
%\bigskip
\textbf{}
 }

\maketitle

%\vspace{-1ex}
%\flushbottom % Makes all text pages the same height

%\maketitle % Print the title and abstract box

%\tableofcontents % Print the contents section

\thispagestyle{empty} % Removes page numbering from the first page

%----------------------------------------------------------------------------------------
%	ARTICLE CONTENTS
%----------------------------------------------------------------------------------------

%\section*{Introduction} % The \section*{} command stops section numbering

%\addcontentsline{toc}{section}{\hspace*{-\tocsep}Introduction} % Adds this section to the table of contents with negative horizontal space equal to the indent for the numbered sections

%\tableofcontents
%\section{ Introduction}

%\section*{Overview}


\begin{multicols}{2}

% \section

Reunimos a seguir uma lista de ações sugeridas para empresas com o intuito de prevenir a transmissão do Coronavírus. Algumas das recomendações são especialmente voltadas para os setores do Comércio e Hotelaria.

\section*{Aspectos Gerais}
\begin{itemize}
\item Promova o conhecimento entre os funcionários e suas famílias sobre a transmissão e prevenção do Coronavírus.
\item Desenvolva políticas personalizadas da organização para reduzir a transmissão e providenciar uma implementação nos mínimos detalhes
\item Certifique-se de que seus funcionários saibam que, mesmo com sintomas leves, não devem estar nos locais de trabalho ou em reuniões pessoais e não serão penalizados por faltarem no trabalho por motivo de doença. Configure um sistema que permita reportar qualquer caso.
\item Garanta que os funcionários tenham plano de saúde adequados, para que não tenham medo de procurar atendimento quando tiverem sintomas, mesmo os mais leves.
\item Mantenha contato com as instalações médicas locais para coordenar testes rápidos e precoces dos funcionários para o Coronavírus
\item Tenha um estoque do essencial (desinfetante para as mãos, álcool, máscaras [1], termômetros infravermelhos na testa sem contato), caso as condições se deteriorem e os funcionários não tenham acesso a esses itens
\item Fortalecer os elos mais fracos da organização reduz a vulnerabilidade
\end{itemize}

\section*{Reuniões, Viagens e Visitantes}
\begin{itemize}
\item Substitua as reuniões presenciais por virtuais
\item Organize-se para que os trabalhadores trabalhem de casa sempre que possível
\item Restrinja as viagens a zonas de maior risco (Vermelho, Laranja e até Amarelo)
\item Elimine viagens não essenciaisAltere as formas de fazer negócios para tornar desnecessárias as viagens aparentemente essenciais.
\item Limite o número des visitantes e defina políticas para consultar e descartar os visitantes com base no status da Zona de Risco onde moram, e nas políticas comerciais de prevenção de Coronavírus. Verifique os sintomas de visitantes logo na sua chegada.
\end{itemize}

\section*{Locais de trabalho}
\begin{itemize}
\item Promova horários de trabalho flexíveis, escalonados e em turnos para diminuir a densidade no local de trabalho. A densidade deve ser reduzida para menos de 50\% da capacidade em um determinado momento.
\item Os empregadores devem solicitar que os funcionários que estão retornando de locais com casos confirmados ou que tenham tido provável contato durante a viagem que se coloquem em quarentena por 14 dias antes de retornar ao escritório. Os empregadores devem acompanhar de perto sua condição de saúde, reportar e procurar atendimento médico.
\item Os pontos de entrada devem ser monitorados com indivíduos portando termômetros infravermelhos que não exijam contato.
\item Meça a temperatura corporal dos funcionários diariamente e forneça a eles máscaras [1] onde a proximidade com os outros não possa ser evitada.
\item Redirecione o tráfego dentro do prédio para promover a lavagem das mãos na entrada e coloque o desinfetante de mãos (álcool gel) na entrada do escritório.
\item Organize os funcionários para que evitem agrupamentos em elevadores. Os elevadores não devem ocupar mais da metade de sua capacidade de carga.Verifique se o espaço de trabalho de cada funcionário está separado por pelo menos 1 metro de distância e que cada espaço de trabalho individual deve ter pelo menos 8 metros quadrados. Para escritórios com grande número de pessoas, essas diretrizes mínimas devem ser aumentadas.
\item Desinfete áreas públicas, os locais com tráfego intenso, e as superfícies frequentemente tocadas.
\item Se o ar condicionado precisar ser usado, desative a recirculação do ar interno. Semanalmente, limpe/desinfete/substitua os principais componentes e filtros.
\item Realize refeições dispersas, mantendo 1 metro de distância entre os funcionários durante as refeições e evite que os funcionários sentem de frente um para o outro. Separe os utensílios e desinfete-os com frequência. Os funcionários da cozinha devem ser verificados com frequência quanto à saúde.
\item Promova a entrega de refeições em vez de sair para comer. Providencie ajuda para solicitar as entregas e um local limpo e seguro para entrega e retirada sem contato de alimentos e sem filas.
\item Leve em conta o percurso dos funcionários até o local de trabalho e desenvolva recomendações, incluindo: evitar o transporte público; higiene cuidadosa, evitando tocar em superfícies, lavando as mãos e usando máscaras [1] em áreas de risco elevado.
\item A delegação de responsabilidades de garantir políticas no local de trabalho sobre a segurança do Coronavírus deve se dar de forma clara e responsabilizante.
\end{itemize}
%
\section*{Varejo, Comércio e Hotelaria}
\begin{itemize}
\item Negócios que envolvem alto contato físico podem ser severamente afetados. Intervenções antecipadas e eficazes podem atenuar, mas não eliminar o risco, a menos que sejam tomadas por toda a sociedade.
\item A importância de garantir que indivíduos com sintomas leves de resfriado não compareçam ao trabalho e nem entrem em contato com outros funcionários não pode nunca ser ignorada.
\item Mantenha um registro claro dos contatos de cada dia, para que, se uma infecção for identificada, a empresa possa notificar imediatamente todos os que foram possivelmente expostos, para que se minimizem os riscos e se mitiguem os danos a funcionários e clientes.
\item Devem ser implementados métodos sem necessidade de contato físico para fazer negócios, como:
  \begin{itemize}
    \item Serviço de entrega e devolução por janelas, para evitar contato pessoal, garantindo espaçamento adequado nas filas
    \item Serviços prestados para que pessoas se mantenham dentro de seus carros (do tipo "Drive-through")
    \item Entregas em domicílio sem contato pessoal
  \end{itemize}
\end{itemize}

% \section

[1] Apesar do uso de máscaras ser debatido, observamos que: (1) A Qualquer pessoa que tenha sintomas leves deve evitar o contato com outras pessoas e usar máscara enquanto estiver em contato público ou privado com outras pessoas. (2) O uso da máscara deve ser aceito em locais públicos para impedir que os doentes hesitem ou se sintam estigmatizados usando uma máscara. (3) Embora as máscaras não garantam segurança para um indivíduo saudável, e sua disponibilidade possa ser limitada devido à maior prioridade nas condições médicas, o uso de máscaras nos casos em que não se pode evitar o contato físico com outras pessoas reduz drasticamente o risco de infecção. (4) Para os indivíduos com alta suscetibilidade à doença (acima de 50 anos ou com condições de saúde preexistentes), assim como para aqueles que estão em áreas de risco elevado, o alto custo de ser infectado justifca o uso da máscara.


\end{multicols}

\vspace{2ex}







% \bibliography{MyCollection.bib}
\bibliography{references.bib}

\end{document}

