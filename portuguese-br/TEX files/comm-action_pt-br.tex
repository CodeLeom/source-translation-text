%\documentclass[twocolumn,journal]{IEEEtran}
\documentclass[onecolumn,journal]{IEEEtran}
\usepackage{amsfonts}
\usepackage{amsmath}
\usepackage{amsthm}
\usepackage{amssymb}
\usepackage{graphicx}
\usepackage[T1]{fontenc}
%\usepackage[english]{babel}
\usepackage{supertabular}
\usepackage{longtable}
\usepackage[usenames,dvipsnames]{color}
\usepackage{bbm}
%\usepackage{caption}
\usepackage{fancyhdr}
\usepackage{breqn}
\usepackage{fixltx2e}
\usepackage{capt-of}
%\usepackage{mdframed}
\setcounter{MaxMatrixCols}{10}
\usepackage{tikz}
\usetikzlibrary{matrix}
\usepackage{endnotes}
\usepackage{soul}
\usepackage{marginnote}
%\newtheorem{theorem}{Theorem}
\newtheorem{lemma}{Lemma}
%\newtheorem{remark}{Remark}
%\newtheorem{error}{\color{Red} Error}
\newtheorem{corollary}{Corollary}
\newtheorem{proposition}{Proposition}
\newtheorem{definition}{Definition}
\newcommand{\mathsym}[1]{}
\newcommand{\unicode}[1]{}
\newcommand{\dsum} {\displaystyle\sum}
\hyphenation{op-tical net-works semi-conduc-tor}
\usepackage{pdfpages}
\usepackage{enumitem}
\usepackage{multicol}
\usepackage[utf8]{inputenc}

\headsep = 5pt
\textheight = 730pt
%\headsep = 8pt %25pt
%\textheight = 720pt %674pt
%\usepackage{geometry}

\bibliographystyle{unsrt}

\usepackage{float}

 \usepackage{xcolor}

\usepackage[framemethod=TikZ]{mdframed}
%%%%%%%FRAME%%%%%%%%%%%
\usepackage[framemethod=TikZ]{mdframed}
\usepackage{framed}
    % \BeforeBeginEnvironment{mdframed}{\begin{minipage}{\linewidth}}
     %\AfterEndEnvironment{mdframed}{\end{minipage}\par}

\usepackage{hyperref}
\hypersetup{
    colorlinks=true,
    linkcolor=blue,
    filecolor=magenta,      
    urlcolor=cyan,
}

%	%\mdfsetup{%
%	%skipabove=20pt,
%	nobreak=true,
%	   middlelinecolor=black,
%	   middlelinewidth=1pt,
%	   backgroundcolor=purple!10,
%	   roundcorner=1pt}

\mdfsetup{%
	outerlinewidth=1,skipabove=20pt,backgroundcolor=yellow!50, outerlinecolor=black,innertopmargin=0pt,splittopskip=\topskip,skipbelow=\baselineskip, skipabove=\baselineskip,ntheorem,roundcorner=5pt}

\mdtheorem[nobreak=true,outerlinewidth=1,%leftmargin=40,rightmargin=40,
backgroundcolor=yellow!50, outerlinecolor=black,innertopmargin=0pt,splittopskip=\topskip,skipbelow=\baselineskip, skipabove=\baselineskip,ntheorem,roundcorner=5pt,font=\itshape]{result}{Result}


\mdtheorem[nobreak=true,outerlinewidth=1,%leftmargin=40,rightmargin=40,
backgroundcolor=yellow!50, outerlinecolor=black,innertopmargin=0pt,splittopskip=\topskip,skipbelow=\baselineskip, skipabove=\baselineskip,ntheorem,roundcorner=5pt,font=\itshape]{theorem}{Theorem}

\mdtheorem[nobreak=true,outerlinewidth=1,%leftmargin=40,rightmargin=40,
backgroundcolor=gray!10, outerlinecolor=black,innertopmargin=0pt,splittopskip=\topskip,skipbelow=\baselineskip, skipabove=\baselineskip,ntheorem,roundcorner=5pt,font=\itshape]{remark}{Remark}

\mdtheorem[nobreak=true,outerlinewidth=1,%leftmargin=40,rightmargin=40,
backgroundcolor=pink!30, outerlinecolor=black,innertopmargin=0pt,splittopskip=\topskip,skipbelow=\baselineskip, skipabove=\baselineskip,ntheorem,roundcorner=5pt,font=\itshape]{quaestio}{Quaestio}

\mdtheorem[nobreak=true,outerlinewidth=1,%leftmargin=40,rightmargin=40,
backgroundcolor=yellow!50, outerlinecolor=black,innertopmargin=5pt,splittopskip=\topskip,skipbelow=\baselineskip, skipabove=\baselineskip,ntheorem,roundcorner=5pt,font=\itshape]{background}{Background}

%TRYING TO INCLUDE Ppls IN TOC
\usepackage{hyperref}


\begin{document}
\title{\color{Brown}  Ação Comunitária e Suporte \\ durante o Surto de COVID-19
\vspace{-0.35ex}}
\author{Naomi Bar-Yam, Chen Shen e Yaneer Bar-Yam \\ New England Complex Systems Institute \\
\vspace{+0.35ex}
\small{\textit{(traduzido por Lucas Pontes})}\\
 \today
  \vspace{-8ex} \\
%\bigskip
\textbf{}
 }

\maketitle

%\vspace{-1ex}
%\flushbottom % Makes all text pages the same height

%\maketitle % Print the title and abstract box

%\tableofcontents % Print the contents section

\thispagestyle{empty} % Removes page numbering from the first page

%----------------------------------------------------------------------------------------
%	ARTICLE CONTENTS
%----------------------------------------------------------------------------------------

%\section*{Introduction} % The \section*{} command stops section numbering

%\addcontentsline{toc}{section}{\hspace*{-\tocsep}Introduction} % Adds this section to the table of contents with negative horizontal space equal to the indent for the numbered sections

%\tableofcontents
%\section{ Introduction}

%\section*{Overview}


\begin{multicols}{2}

Como as ordens de isolamento em locais (quarentenas) são necessárias para interromper o COVID-19, haverá interrupções no trabalho e no serviço que afetarão os indivíduos de maneira diferente e até séria. Também é importante abordar o isolamento social e tudo o que isso implica. Família, amigos e comunidade são sistemas de apoio essenciais.

Existem vários aspectos de como os indivíduos podem desenvolver seu envolvimento com a comunidade e como uma comunidade organizada pode desenvolver apoio mútuo:

\begin{itemize}
    \item Os indivíduos podem entrar em contato com seus amigos e familiares, perguntar se necessitam de ajuda, e que ajuda podem oferecer. Acompanhe quem pode ajudar você, e com o quê, e passe as informações para outras pessoas que precisam. Decida com quem você deseja entrar em contato regularmente e coloque esses textos, chamadas ou videochamadas em sua agenda. Incentive outras pessoas a fazer o mesmo.
    \item A comunidade pode criar um sistema de amigos. Pequenos grupos de 2 ou 3 indivíduos ou famílias são um “Buddy Pod” que se comunica remotamente duas vezes por semana. Mesmo que todo mundo esteja e permaneça livre do COVID 19, haverá preocupações, problemas etc. Exemplos: colete pedidos de compras de amigos e alterne quem vai à loja, usando o serviço de entrega sem contato. Ajude aqueles que não estão acostumados a fazer pedidos on-line. Preste atenção ao que são as necessidades e seja criativo.
    \item Priorize, com cautela extra, ajudar os membros mais velhos da comunidade que não moram com seus filhos a configurar telecomunicações e comunicação virtual, especialmente vídeo. Deve-se tomar muito cuidado para evitar qualquer contato físico direto ou proximidade, pois os idosos são particularmente vulneráveis a doenças graves e morte.
    \item No Buddy Pod, os indivíduos devem reconhecer as oportunidades no tempo disponível e perguntar um ao outro o que estão interessados em fazer com ele. Por exemplo, “Se eu tivesse algum tempo livre, sempre quis aprender ...” e ajude / ative / promova um ao outro para realizá-los.
    \item Oportunidade de interação entre gerações: nem todo mundo é capaz de usar a Internet ou cozinhar, consertar e até ler. Torne os grupos e equipes de amigos diversos, para que a ajuda mútua seja mais significativa e divertida. Pessoas com filhos adultos podem adorar a ajuda remota para a lição de casa e ler com os membros mais jovens.
    \item Dependendo do tamanho da comunidade, subgrupos maiores de 5 a 10 pods de amigos, por exemplo, podem formar um "Grupo de amigos" que estará disponível um para o outro como recurso adicional para debater e trabalhar em conjunto para resolver problemas maiores do que um "Buddy Pod" pode suportar. Se um amigo ficar doente, com o COVID 19 ou qualquer outra coisa, qual a melhor forma de apoiar e permanecer seguro? Que outros recursos podem ser obtidos?
    \item Que suprimentos a comunidade pode comprar em grandes quantidades e ter disponível para os membros? Alimentos não perecíveis, luvas, lenços de limpeza, toalhas de papel, lenços de papel, e outros itens que tem esgotado em lojas (virtuais e físicas), até que estejam novamente disponíveis.
    \item As instituições comunitárias podem ser criativas ao fornecer remotamente as atividades que normalmente são feitas in situ. Por exemplo, as universidades estão realizando aulas on-line. Casas de culto estão transmitindo cultos e aulas religiosas. Clubes de livros, pontes e tricô se encontrariam no Zoom?
    \item Em lugares de clima quente, planeje atividades ao ar livre para pequenos grupos que permitam aos participantes manter distância (mais de um metro e oitenta), mas que gostam de estar juntos no mesmo espaço externo.
    \item Eventos - aniversários e aniversários, casamentos, formaturas, acolher novos bebês e mortes - os momentos em que nos reunimos para comemorar, lamentar e dar apoio estão mudando agora. Os abraços terão que esperar, mas como a comunidade pode apoiar a celebração juntos?
    \item Suporte mútuo no local de trabalho: algumas pessoas podem fazer seu trabalho em casa, outras não. Em alguns locais de trabalho, as pessoas com um trabalho específico precisam estar presentes, mas não todas de uma vez, como por exemplo, equipes médicas, supermercados, farmácias, asilos. Trabalhe em conjunto para criar um cronograma seguro de trabalho, e garanta que, se uma pessoa ou equipe (se for assim que for configurada) ficar doente, o local de trabalho ainda possa funcionar, e que ninguém fique exausto.
    \item Os pods de amigos também podem trabalhar em direção à formação de Espaços Seguros, abrindo a possibilidade de reunir-se pessoalmente (para mais detalhes sobre Espaços Seguros, leia o documento \href{https://github.com/necsi/source-translation-text/raw/master/portuguese-br/pdf/Family_port-br.pdf}{"Diretrizes para Famílias"}.
    \item Com base no que for possível para a comunidade, os indivíduos e a comunidade devem procurar maneiras de apoiar outras comunidades localmente ou encontrar “comunidades irmãs” globalmente, para compartilhar informações ou colaborar em projetos.
    \item Como a pandemia é global, esta é uma oportunidade de conhecer outras pessoas on-line da comunidade global, falando outro idioma no qual talvez você saiba falar um pouco ou nada, compartilhar a experiência do isolamento, aprender o idioma um do outro, e fornecer e aprender dicas de segurança e prevenção entre si.

\end{itemize}


\end{multicols}

\vspace{2ex}







% \bibliography{MyCollection.bib}
\bibliography{references.bib}

\end{document}
