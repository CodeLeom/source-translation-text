%\documentclass[twocolumn,journal]{IEEEtran}
\documentclass[onecolumn,journal]{IEEEtran}
\usepackage{amsfonts}
\usepackage{amsmath}
\usepackage{amsthm}
\usepackage{amssymb}
\usepackage{graphicx}
\usepackage[T1]{fontenc}
%\usepackage[english]{babel}
\usepackage{supertabular}
\usepackage{longtable}
\usepackage[usenames,dvipsnames]{color}
\usepackage{bbm}
%\usepackage{caption}
\usepackage{fancyhdr}
\usepackage{breqn}
\usepackage{fixltx2e}
\usepackage{capt-of}
%\usepackage{mdframed}
\setcounter{MaxMatrixCols}{10}
\usepackage{tikz}
\usetikzlibrary{matrix}
\usepackage{endnotes}
\usepackage{soul}
\usepackage{marginnote}
%\newtheorem{theorem}{Theorem}
\newtheorem{lemma}{Lemma}
%\newtheorem{remark}{Remark}
%\newtheorem{error}{\color{Red} Error}
\newtheorem{corollary}{Corollary}
\newtheorem{proposition}{Proposition}
\newtheorem{definition}{Definition}
\newcommand{\mathsym}[1]{}
\newcommand{\unicode}[1]{}
\newcommand{\dsum} {\displaystyle\sum}
\hyphenation{op-tical net-works semi-conduc-tor}
\usepackage{pdfpages}
\usepackage{enumitem}
\usepackage{multicol}
\usepackage[utf8]{inputenc}


\headsep = 5pt
\textheight = 730pt
%\headsep = 8pt %25pt
%\textheight = 720pt %674pt
%\usepackage{geometry}

\bibliographystyle{unsrt}

\usepackage{float}

 \usepackage{xcolor}

\usepackage[framemethod=TikZ]{mdframed}
%%%%%%%FRAME%%%%%%%%%%%
\usepackage[framemethod=TikZ]{mdframed}
\usepackage{framed}
    % \BeforeBeginEnvironment{mdframed}{\begin{minipage}{\linewidth}}
     %\AfterEndEnvironment{mdframed}{\end{minipage}\par}


%	%\mdfsetup{%
%	%skipabove=20pt,
%	nobreak=true,
%	   middlelinecolor=black,
%	   middlelinewidth=1pt,
%	   backgroundcolor=purple!10,
%	   roundcorner=1pt}

\mdfsetup{%
	outerlinewidth=1,skipabove=20pt,backgroundcolor=yellow!50, outerlinecolor=black,innertopmargin=0pt,splittopskip=\topskip,skipbelow=\baselineskip, skipabove=\baselineskip,ntheorem,roundcorner=5pt}

\mdtheorem[nobreak=true,outerlinewidth=1,%leftmargin=40,rightmargin=40,
backgroundcolor=yellow!50, outerlinecolor=black,innertopmargin=0pt,splittopskip=\topskip,skipbelow=\baselineskip, skipabove=\baselineskip,ntheorem,roundcorner=5pt,font=\itshape]{result}{Result}


\mdtheorem[nobreak=true,outerlinewidth=1,%leftmargin=40,rightmargin=40,
backgroundcolor=yellow!50, outerlinecolor=black,innertopmargin=0pt,splittopskip=\topskip,skipbelow=\baselineskip, skipabove=\baselineskip,ntheorem,roundcorner=5pt,font=\itshape]{theorem}{Theorem}

\mdtheorem[nobreak=true,outerlinewidth=1,%leftmargin=40,rightmargin=40,
backgroundcolor=gray!10, outerlinecolor=black,innertopmargin=0pt,splittopskip=\topskip,skipbelow=\baselineskip, skipabove=\baselineskip,ntheorem,roundcorner=5pt,font=\itshape]{remark}{Remark}

\mdtheorem[nobreak=true,outerlinewidth=1,%leftmargin=40,rightmargin=40,
backgroundcolor=pink!30, outerlinecolor=black,innertopmargin=0pt,splittopskip=\topskip,skipbelow=\baselineskip, skipabove=\baselineskip,ntheorem,roundcorner=5pt,font=\itshape]{quaestio}{Quaestio}

\mdtheorem[nobreak=true,outerlinewidth=1,%leftmargin=40,rightmargin=40,
backgroundcolor=yellow!50, outerlinecolor=black,innertopmargin=5pt,splittopskip=\topskip,skipbelow=\baselineskip, skipabove=\baselineskip,ntheorem,roundcorner=5pt,font=\itshape]{background}{Background}

%TRYING TO INCLUDE Ppls IN TOC
\usepackage{hyperref}


\begin{document}
\title{\color{Brown} Diretrizes Essenciais Básicas contra o Coronavírus \\
\vspace{-0.35ex}}
\author{Chen Shen e Yaneer Bar-Yam \\ New England Complex Systems Institute \\
\vspace{+0.35ex}
\small{\textit{(traduzido por Lucas Pontes})}\\
 \today
  \vspace{-14ex} \\


\bigskip
\bigskip

\textbf{}
 }

\maketitle


\flushbottom % Makes all text pages the same height

%\maketitle % Print the title and abstract box

%\tableofcontents % Print the contents section

\thispagestyle{empty} % Removes page numbering from the first page

%----------------------------------------------------------------------------------------
%	ARTICLE CONTENTS
%----------------------------------------------------------------------------------------

%\section*{Introduction} % The \section*{} command stops section numbering

%\addcontentsline{toc}{section}{\hspace*{-\tocsep}Introduction} % Adds this section to the table of contents with negative horizontal space equal to the indent for the numbered sections

%\tableofcontents
%\section{ Introduction}
\renewcommand{\thefootnote}{\fnsymbol{footnote}}




\begin{multicols}{2}


\section*{I. Para todas as pessoas}
\begin{itemize}
\item Reconheça que a situação não é normal, e que o comportamento rotineiro deve ser modificado para sobreviver, proteger e sustentar a vida.
\item Em 80\% dos casos a doença é leve, 20\% grave requer hospitalização, 10\% em unidades de terapia intensiva e 2-4\% de mortes. Mesmo os casos leves são contagiosos e podem levar a casos graves para outras pessoas ao espalharem infecções, todos têm a responsabilidade de proteger a família, os amigos e a sociedade de mais danos.
\item Minimize o contato, mantenha ações essenciais e normais sempre que possível, e maximize a ajuda a outras pessoas.
\item A resposta ideal imediata pode não ser possível, mas todas as ações que reduzem a exposição à infecção reduzem o risco e o surto, e as melhorias podem ocorrer ao longo do tempo e através da colaboração.
\end{itemize}


\section*{II. Auto-isolamento}
\begin{itemize}
\item Fique a 2 m de distância dos outros, e não toque em superfícies inseguras compartilhadas / públicas.
\item Ajude outras pessoas localmente sem contato físico, e engaje outras pessoas on-line.
\item Use sabão, desinfetante, luvas e sacos plásticos para reduzir contato.
\item Mantenha-se informado sobre os sintomas típicos, instruções locais para teste/tratamento, e esteja preparado para agir o mais rápido possível quando qualquer sintoma se manifestar.
\item Mantenha contato constantemente com um amigo/parente confiável, trocando informações sobre sintomas, caso os sinta (há casos em que os sintomas ocorrem muito rapidamente).
\end{itemize}

\section*{III. Para famílias e amigos}
\begin{itemize}
\item Compartilhe espaços seguros com familiares / amigos que concordam em se isolar dos outros.
\item Coordene-se com outras pessoas em espaços seguros para ajuda mútua.
\item Indivíduos com sintomas ou identificação positiva de caso devem isolar-se separadamente da família e dos amigos.
\end{itemize}

\section*{IV. Comunidade}
\begin{itemize}
\item Identifique zonas e, se possível, proteja os limites, permitindo apenas tráfego essencial.
\item Organize entregas de produtos e suprimentos sem necessidade de contato físico.
\item Certifique os indivíduos seguros. Rastreie os membros que apresentarem sintomas, teste-os e isole-os.
\item Forneça atendimento sem contato para os que estão isolados.
\item Expanda instalações e serviços compartilhados dos espaços seguros.
\end{itemize}

\section*{V. Empresas}
\begin{itemize}
\item Maximize a possibilidade de trabalho em casa para permitir o auto-isolamento e promover espaços seguros.
\item Mantenha funções essenciais e reduza o impacto em todas as funções usando a estratégia de Espaços Seguros.
\end{itemize}

\section*{VI. Sistemas de saúde}
\begin{itemize}
\item Divida as instalações de saúde em áreas para evitar contágio total. Configure grandes espaços temporários separados para diferentes níveis de atendimento.
\end{itemize}

\section*{VII Comunidades/Governos/Empresas}
\begin{itemize}
\item Organize testes rápidos para identificar indivíduos seguros e indivíduos para isolamento.
\item Isole comunidades com transmissão ativa, e auxilie-as em serviços essenciais.
\end{itemize}


\end{multicols}



% \bibliography{MyCollection.bib}


\end{document}

