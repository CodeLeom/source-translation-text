%\documentclass[twocolumn,journal]{IEEEtran}
\documentclass[onecolumn,journal]{IEEEtran}
\usepackage{amsfonts}
\usepackage{amsmath}
\usepackage{amsthm}
\usepackage{amssymb}
\usepackage{graphicx}
\usepackage[T1]{fontenc}
%\usepackage[english]{babel}
\usepackage{supertabular}
\usepackage{longtable}
\usepackage[usenames,dvipsnames]{color}
\usepackage{bbm}
%\usepackage{caption}
\usepackage{fancyhdr}
\usepackage{breqn}
\usepackage{fixltx2e}
\usepackage{capt-of}
%\usepackage{mdframed}
\setcounter{MaxMatrixCols}{10}
\usepackage{tikz}
\usetikzlibrary{matrix}
\usepackage{endnotes}
\usepackage{soul}
\usepackage{marginnote}
%\newtheorem{theorem}{Theorem}
\newtheorem{lemma}{Lemma}
%\newtheorem{remark}{Remark}
%\newtheorem{error}{\color{Red} Error}
\newtheorem{corollary}{Corollary}
\newtheorem{proposition}{Proposition}
\newtheorem{definition}{Definition}
\newcommand{\mathsym}[1]{}
\newcommand{\unicode}[1]{}
\newcommand{\dsum} {\displaystyle\sum}
\hyphenation{op-tical net-works semi-conduc-tor}
\usepackage{pdfpages}
\usepackage{enumitem}
\usepackage{multicol}
\usepackage[utf8]{inputenc}

\headsep = 5pt
\textheight = 730pt
%\headsep = 8pt %25pt
%\textheight = 720pt %674pt
%\usepackage{geometry}

\bibliographystyle{unsrt}

\usepackage{float}

% \usepackage{xcolor}

\usepackage[framemethod=TikZ]{mdframed}
%%%%%%%FRAME%%%%%%%%%%%
\usepackage[framemethod=TikZ]{mdframed}
\usepackage{framed}
    % \BeforeBeginEnvironment{mdframed}{\begin{minipage}{\linewidth}}
     %\AfterEndEnvironment{mdframed}{\end{minipage}\par}


%	%\mdfsetup{%
%	%skipabove=20pt,
%	nobreak=true,
%	   middlelinecolor=black,
%	   middlelinewidth=1pt,
%	   backgroundcolor=purple!10,
%	   roundcorner=1pt}

\mdfsetup{%
	outerlinewidth=1,skipabove=20pt,backgroundcolor=yellow!50, outerlinecolor=black,innertopmargin=0pt,splittopskip=\topskip,skipbelow=\baselineskip, skipabove=\baselineskip,ntheorem,roundcorner=5pt}

\mdtheorem[nobreak=true,outerlinewidth=1,%leftmargin=40,rightmargin=40,
backgroundcolor=yellow!50, outerlinecolor=black,innertopmargin=0pt,splittopskip=\topskip,skipbelow=\baselineskip, skipabove=\baselineskip,ntheorem,roundcorner=5pt,font=\itshape]{result}{Result}


\mdtheorem[nobreak=true,outerlinewidth=1,%leftmargin=40,rightmargin=40,
backgroundcolor=yellow!50, outerlinecolor=black,innertopmargin=0pt,splittopskip=\topskip,skipbelow=\baselineskip, skipabove=\baselineskip,ntheorem,roundcorner=5pt,font=\itshape]{theorem}{Theorem}

\mdtheorem[nobreak=true,outerlinewidth=1,%leftmargin=40,rightmargin=40,
backgroundcolor=gray!10, outerlinecolor=black,innertopmargin=0pt,splittopskip=\topskip,skipbelow=\baselineskip, skipabove=\baselineskip,ntheorem,roundcorner=5pt,font=\itshape]{remark}{Remark}

\mdtheorem[nobreak=true,outerlinewidth=1,%leftmargin=40,rightmargin=40,
backgroundcolor=pink!30, outerlinecolor=black,innertopmargin=0pt,splittopskip=\topskip,skipbelow=\baselineskip, skipabove=\baselineskip,ntheorem,roundcorner=5pt,font=\itshape]{quaestio}{Quaestio}

\mdtheorem[nobreak=true,outerlinewidth=1,%leftmargin=40,rightmargin=40,
backgroundcolor=yellow!50, outerlinecolor=black,innertopmargin=5pt,splittopskip=\topskip,skipbelow=\baselineskip, skipabove=\baselineskip,ntheorem,roundcorner=5pt,font=\itshape]{background}{Background}

%TRYING TO INCLUDE Ppls IN TOC
\usepackage{hyperref}


\begin{document}
\title{\color{Brown}  Diretrizes especiais para profissionais de saúde durante a pandemia da COVID-19*
\vspace{-0.35ex}}
\author{Paige Voltaire, Chen Shen e Yaneer Bar-Yam \\ New England Complex Systems Institute \\
\vspace{+0.35ex}
\small{\textit{(traduzido por Lucas Pontes})}\\
 \today
  \vspace{-8ex} \\
%\bigskip
\textbf{}
 }

\maketitle

%\vspace{-1ex}
%\flushbottom % Makes all text pages the same height

%\maketitle % Print the title and abstract box

%\tableofcontents % Print the contents section

\thispagestyle{empty} % Removes page numbering from the first page

%----------------------------------------------------------------------------------------
%	ARTICLE CONTENTS
%----------------------------------------------------------------------------------------

%\section*{Introduction} % The \section*{} command stops section numbering

%\addcontentsline{toc}{section}{\hspace*{-\tocsep}Introduction} % Adds this section to the table of contents with negative horizontal space equal to the indent for the numbered sections

%\tableofcontents
%\section{ Introduction}

%\section*{Overview}


\begin{multicols}{2}

Médicos, enfermeiros, médicos e assistentes médicos: vocês estão na linha de frente de uma guerra contra a pandemia do COVID-19. Devido à sua posição e importância, contamos com você para que tenha o melhor desempenho, e para que siga algumas diretrizes simples, porém rigorosas, para retardar a propagação desse vírus - salvando ainda mais vidas do que você já está salvando nesse processo.

\begin{itemize}
    \item Obviamente, use EPI adequado quando disponível. Revise os métodos apropriados para colocar e remover (vestir e retirar) os equipamentos com segurança. Observe que o risco mais alto ocorre durante a remoção do equipamento. Reutilize o EPI se solicitado. Existem vários métodos de desinfecção aprovados pelo CDC, incluindo luz UV e ozônio.  Acompanhe as ações da liderança do hospital e do Diretor Médico nos processos de esterilização/desinfecção de EPI para reutilização. Suprimentos, equipamentos e suporte estão a caminho, se ainda não estiverem lá.
    \item Horário de trabalho/sono: Como nação e sociedade, confiamos em você para poder desempenhar suas funções médicas da melhor maneira possível. Para tornar isso possível, você precisa de sono adequado. Isto é uma necessidade absoluta. Tente entrar em contato com seus Diretores Médicos, Enfermeiros Responsáveis, Enfermeiros de Piso e/ou outra liderança para estabelecer e implementar períodos obrigatórios de sono e descanso para cada trabalhador ou equipe. A estratégia recomendada neste cenário de emergência é de no máximo 18 horas de atendimento ao paciente, com um mínimo de 12 horas de descanso ininterrupto.
    \item As vias de acesso dentro do hospital, incluindo corredores e elevadores, não são espaços seguros e requerem um nível de proteção. Ter áreas seguras designadas para os funcionários é extremamente útil para chegada, partida e pausas. Quando a pressão é muito alta, o desafio de entrar e sair dos EPIs para beber, comer e descansar nos banheiros deve ser reduzido o máximo possível.
    \item Devem ser padronizados os procedimentos para a colocação e retirada do equipamento de proteção. Zonas reservadas devem ser identificadas. Faça fluxogramas de diferentes zonas, forneça espelhos de corpo inteiro e observe estritamente as rotas a pé.
    \item Não podemos ter trabalhadores da área da saúde privados de sono, imunocomprometidos, trabalhando excessivamente ou desmoralizados. Quando essas cenários surgem, erros são cometidos, como de dosagem de medicamentos; as pessoas discordam e se deterioram; o atendimento ao paciente é prejudicado; e os trabalhadores da saúde adoecem e podem rapidamente se tornar pacientes. Isso pode levar a um colapso completo do sistema hospitalar local. Encontrar tempo para relaxar e dormir beneficiará você, seus pacientes, colegas de trabalho e qualquer outra pessoa dentro e fora do Hospital.
    \item O tempo de trabalho consecutivo deve ser reduzido ainda mais quando houver membros adicionais para colaborar na batalha. Um caso anedótico: em Wuhan, quando chegaram médicos/enfermeiros de outras regiões, os médicos passaram a trabalhar 8h por dia e as enfermeiras 6h, por conta da exaustão e intensidade do trabalho. Ações heroicas, embora louváveis, levam a taxas de mortalidade mais altas. Portanto, ordens de descanso obrigatórias são essenciais.
    \item Distanciamento social: Devido a essa situação única, devemos aconselhar (infelizmente) que você absolutamente NÃO volte para casa, onde entraria em contato com entes queridos e familiares. Você passará a maior parte do tempo trabalhando em uma nuvem de COVID-19 e outros patógenos infecciosos desagradáveis. No momento, é muito arriscado agir como numa rotina normal. O contato desnecessário é altamente desaconselhável. No momento, é provável que você possa transmitir esse vírus a outras pessoas da sua família ou grupo familiar, e cada um deles pode se espalhar para outros grupos, regiões e assim por diante. Isso facilmente torna todos os outros métodos que estamos usando menos eficazes em "esmagar a curva". Se possível, mantenha-se afastado de todas as pessoas que não são necessárias para o exercício de sua profissão.
    \item Considere hospedar-se em hotéis, camas/quartos de hospital não utilizados ou outros alojamentos disponíveis. Se você mora sozinho, ou conhece alguém que mora, você pode perguntar se você e alguns com quem trabalha podem ficar lá por algum tempo. Se precisar ir para casa, isole-se dos outros, use uma máscara, tome banho com água quente e sabão e coloque suas roupas sujas em um zíper ou saco de lixo. Estamos trabalhando para fornecer a você um espaço extra próprio, sem nenhum custo.
    \item Organize equipes, tipicamente de 3 a 5 pessoas, para facilitar os trabalhos, para facilitar a ampliação do atendimento à medida que o número de casos aumentar, para fornecer atendimento de alta qualidade ao paciente, e para apoio mútuo por sinais de doença, lesão ou excesso de trabalho. Exemplos de equipes adaptadas para esta crise: equipes de intubação, equipes de pronto atendimento, equipes extras de caso cardíaco.
    \item As instituições de saúde devem limitar o acesso de pessoas de fora (visitantes, famílias, prestadores de serviços) para reduzir a exposição a eles e à equipe. Os profissionais de saúde em risco que não trabalham diretamente com os pacientes podem ajudar, assim como a equipe de comunicação, a alcançar as famílias, permitindo que as equipes clínicas se concentrem no trabalho clínico.
    
\end{itemize}

*Revisado e editado pelo Dr. Christian DePaola e pela Dra. Margit Kaufman.

\end{multicols}

*Revisado e editado pelo Dr. Christian DePaola e pela Dra. Margit Kaufman.

% \bibliography{MyCollection.bib}
\bibliography{references.bib}

\end{document}
