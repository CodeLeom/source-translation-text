%\documentclass[twocolumn,journal]{IEEEtran}
\documentclass[onecolumn,journal]{IEEEtran}
\usepackage{amsfonts}
\usepackage{amsmath}
\usepackage{amsthm}
\usepackage{amssymb}
\usepackage{graphicx}
\usepackage[T1]{fontenc}
%\usepackage[english]{babel}
\usepackage{supertabular}
\usepackage{longtable}
\usepackage[usenames,dvipsnames]{color}
\usepackage{bbm}
%\usepackage{caption}
\usepackage{fancyhdr}
\usepackage{breqn}
\usepackage{fixltx2e}
\usepackage{capt-of}
%\usepackage{mdframed}
\setcounter{MaxMatrixCols}{10}
\usepackage{tikz}
\usetikzlibrary{matrix}
\usepackage{endnotes}
\usepackage{soul}
\usepackage{marginnote}
%\newtheorem{theorem}{Theorem}
\newtheorem{lemma}{Lemma}
%\newtheorem{remark}{Remark}
%\newtheorem{error}{\color{Red} Error}
\newtheorem{corollary}{Corollary}
\newtheorem{proposition}{Proposition}
\newtheorem{definition}{Definition}
\newcommand{\mathsym}[1]{}
\newcommand{\unicode}[1]{}
\newcommand{\dsum} {\displaystyle\sum}
\hyphenation{op-tical net-works semi-conduc-tor}
\usepackage{pdfpages}
\usepackage{enumitem}
\usepackage{multicol}
\usepackage[utf8]{inputenc}


\headsep = 5pt
\textheight = 730pt
%\headsep = 8pt %25pt
%\textheight = 720pt %674pt
%\usepackage{geometry}

\bibliographystyle{unsrt}

\usepackage{float}

 \usepackage{xcolor}
 
 \usepackage[brazilian]{babel}

\usepackage[framemethod=TikZ]{mdframed}
%%%%%%%FRAME%%%%%%%%%%%
\usepackage[framemethod=TikZ]{mdframed}
\usepackage{framed}
    % \BeforeBeginEnvironment{mdframed}{\begin{minipage}{\linewidth}}
     %\AfterEndEnvironment{mdframed}{\end{minipage}\par}


%	%\mdfsetup{%
%	%skipabove=20pt,
%	nobreak=true,
%	   middlelinecolor=black,
%	   middlelinewidth=1pt,
%	   backgroundcolor=purple!10,
%	   roundcorner=1pt}

\mdfsetup{%
	outerlinewidth=1,skipabove=20pt,backgroundcolor=yellow!50, outerlinecolor=black,innertopmargin=0pt,splittopskip=\topskip,skipbelow=\baselineskip, skipabove=\baselineskip,ntheorem,roundcorner=5pt}

\mdtheorem[nobreak=true,outerlinewidth=1,%leftmargin=40,rightmargin=40,
backgroundcolor=yellow!50, outerlinecolor=black,innertopmargin=0pt,splittopskip=\topskip,skipbelow=\baselineskip, skipabove=\baselineskip,ntheorem,roundcorner=5pt,font=\itshape]{result}{Result}


\mdtheorem[nobreak=true,outerlinewidth=1,%leftmargin=40,rightmargin=40,
backgroundcolor=yellow!50, outerlinecolor=black,innertopmargin=0pt,splittopskip=\topskip,skipbelow=\baselineskip, skipabove=\baselineskip,ntheorem,roundcorner=5pt,font=\itshape]{theorem}{Theorem}

\mdtheorem[nobreak=true,outerlinewidth=1,%leftmargin=40,rightmargin=40,
backgroundcolor=gray!10, outerlinecolor=black,innertopmargin=0pt,splittopskip=\topskip,skipbelow=\baselineskip, skipabove=\baselineskip,ntheorem,roundcorner=5pt,font=\itshape]{remark}{Remark}

\mdtheorem[nobreak=true,outerlinewidth=1,%leftmargin=40,rightmargin=40,
backgroundcolor=pink!30, outerlinecolor=black,innertopmargin=0pt,splittopskip=\topskip,skipbelow=\baselineskip, skipabove=\baselineskip,ntheorem,roundcorner=5pt,font=\itshape]{quaestio}{Quaestio}

\mdtheorem[nobreak=true,outerlinewidth=1,%leftmargin=40,rightmargin=40,
backgroundcolor=yellow!50, outerlinecolor=black,innertopmargin=5pt,splittopskip=\topskip,skipbelow=\baselineskip, skipabove=\baselineskip,ntheorem,roundcorner=5pt,font=\itshape]{background}{Background}

%TRYING TO INCLUDE Ppls IN TOC
\usepackage{hyperref}


\begin{document}
\title{\color{Brown} Questões para Triagem e Segurança dos Empregados acerca do COVID-19 \\
\vspace{-0.35ex}}
\author{Naomi Bar-Yam, Chen Shen, e Yaneer Bar-Yam \\ New England Complex Systems Institute \\
\vspace{+0.35ex}
\small{\textit{(traduzido por Guilhermo Costa})}\\
 \today
  \vspace{-14ex} \\


\bigskip
\bigskip

\textbf{}
 }

\maketitle


\flushbottom % Makes all text pages the same height

%\maketitle % Print the title and abstract box

%\tableofcontents % Print the contents section

\thispagestyle{empty} % Removes page numbering from the first page

%----------------------------------------------------------------------------------------
%	ARTICLE CONTENTS
%----------------------------------------------------------------------------------------

%\section*{Introduction} % The \section*{} command stops section numbering

%\addcontentsline{toc}{section}{\hspace*{-\tocsep}Introduction} % Adds this section to the table of contents with negative horizontal space equal to the indent for the numbered sections

%\tableofcontents
%\section{ Introduction}
\renewcommand{\thefootnote}{\fnsymbol{footnote}}




\begin{multicols}{2}

Os empregadores que estão se esforçando para proteger seus consumidores e empregados, especialmente aqueles que estão prestando serviços essenciais durante uma quarentena, devem envolver seus funcionários para que analisem os riscos de serem infectados, com o objetivo de protegê-los individual e coletivamente. Essas informações podem contribuir para a avaliação de como organizar os espaços de trabalho, e de como identificar quais empregados cumprem funções que requerem contato físico quando é necessário. Isso é particularmente relevante para serviços essenciais como supermercados, farmácias, mercearias, e provedores de assistência médica, assim como instituições sujeitas a alto risco, como casas de repouso, dormitórios, asilos, centros de reabilitação, enfermaria psiquiátrica, e prisões. Isso também é relevante em geral para qualquer companhia que se esforça para proteger seus empregados, estejam eles trabalhando em casa ou ociosos, até as atividades da empresa serem restauradas.

Diretrizes gerais para organizações incluem:

\begin{itemize}
\item
  Maximizar o trabalho em casa para permitir o auto-isolamento e promover espaços seguros.
\item
  Manter funções essenciais e reduzir o impacto em todas as funções por meio de espaços de trabalho seguros.
\end{itemize}

\textit{\textbf{Nota do tradutor:} para aprender mais sobre o conceito de espaços seguros, veja o guia  \href{https://github.com/necsi/source-translation-text/raw/master/portuguese-br/pdf/Family_port-br.pdf}{"Diretrizes para Famílias"}.}

Estender as práticas de segurança da empresa a cada um dos funcionários diminui o risco dos mesmos e o impacto na organização. Perguntar aos empregadores o quão seguro seus ambientes estão fora do trabalho é essencial para a segurança do local de trabalho, e o quão sustentável sua contribuição será durante e após esse tempo crítico. A segurança desses ambientes é afetada por possíveis exposições durante o período dos últimos 14 dias. Ações que podem ser tomadas incluem:

\begin{itemize}
\item Encorajar indivíduos a implementarem práticas de Espaços Seguros, incluindo o isolamento em relação às pessoas em risco.
\item Incentivar os indivíduos a manterem atividades saudáveis para reduzir seus fatores de risco.
\item Caso ocorra uma possível exposição, providencie uma maneira segura para que o indivíduo em questão possa se auto-isolar de maneira segura para assegurar que o mesmo não foi infectado, ou, caso o indivíduo  tenha sido infectado, para que seus sintomas se manifestem.
\item Em casos onde os riscos são altos, providenciar moradia separada para empregados que estão bem no momento para evitar que estes sejam infectados, ou que acabem infectando seus colegas de quarto ou familiares por um período de 14 dias:
\end{itemize}

  \begin{itemize}
  \tightlist
  \item especialmente para empregados em instituições de alto risco.
  \item Especialmente no caso em que os empregados vivem com colegas de quarto ou familiares em risco de serem infectados (por exemplo, porque estes estavam ou estão trabalhando em locais de trabalho inseguros).
  \item Epecialmente no caso em que os empregados vivem com colegas de quarto ou familiares que são idosos ou tem condições médicas pré-existentes que elevam os riscos de uma infecção.
  \item Especialmente se o empregado trabalha em uma profissão de alto risco, como trabalhadores do ramo da saúde e trabalhadores essenciais de alto contato no ramo de serviços, como trabalhadores em supermercados, mercearias, e farmácias.
  \end{itemize}

\section*{Questionário de triagem de segurança para
empregadores}

A seguir, há uma lista de perguntas úteis para começar a determinar o nível de risco dos empregados:

\begin{enumerate}
\def\labelenumi{\arabic{enumi}.}
\tightlist
\item Onde você mora?

  \begin{enumerate}
  \def\labelenumii{\alph{enumii}.}
  \tightlist
  \item Casa, duplex, apartamento
  \item Você tem uma entrada própria separada?

    \begin{enumerate}
    \def\labelenumiii{\roman{enumiii}.}
    \tightlist
    \item Portaria comum/lobby
    \item Elevador
    \end{enumerate}
    
  \end{enumerate}
  
\item Quantas pessoas moram na sua casa?

  \begin{enumerate}
  \def\labelenumii{\alph{enumii}.}
  \tightlist
  \item O que elas tem feito nesses dias?

    \begin{enumerate}
    \def\labelenumiii{\roman{enumiii}.}
    \tightlist
    \item Trabalhando fora de casa?
    \item Na escola ou em alguma outra atividade coletiva?
    \item Em casa?
    \end{enumerate}
    
    
  \item Você ou alguém com quem você vive viajou recentemente?

    \begin{enumerate}
    \def\labelenumiii{\roman{enumiii}.}
    \tightlist
    \item Para onde?
    \item Qual foi o meio de transporte utilizado?
    \end{enumerate}
    
  \end{enumerate}
  
\item Como você tem se socializado agora? E quanto?

  \begin{enumerate}
  \def\labelenumii{\alph{enumii}.}
  \tightlist
  \item Você está se isolando socialmente?
  \item Você está recebendo convidados ou sendo convidado?
  \end{enumerate}
  
  
\item O que você sabe sobre medidas de distanciamento social? E quais delas você está praticando?

  \begin{enumerate}
  \def\labelenumii{\alph{enumii}.}
  \tightlist
  \item Distância de 2 metros entre as pessoas
  \item Lavagem das mãos
  \item Não tocar em superfícies e mantê-las limpas
  \end{enumerate}
  
\item Como você tem obtido seus alimentos? Compras no mercado, entregas em casa?
\item Você tem animais de estimação que devem passear fora de casa?
\item  Você ou alguém que mora com você tem algum fator de risco (seja ele uma condição médica, relacionado à idade, ou algum outro) relevante à infecção por COVID-19?

\end{enumerate}


\end{multicols}


% \bibliography{MyCollection.bib}


\end{document}
