%\documentclass[twocolumn,journal]{IEEEtran}
\documentclass[onecolumn,journal]{IEEEtran}
\usepackage{amsfonts}
\usepackage{amsmath}
\usepackage{amsthm}
\usepackage{amssymb}
\usepackage{graphicx}
\usepackage[T1]{fontenc}
%\usepackage[english]{babel}
\usepackage{supertabular}
\usepackage{longtable}
\usepackage[usenames,dvipsnames]{color}
\usepackage{bbm}
%\usepackage{caption}
\usepackage{fancyhdr}
\usepackage{breqn}
\usepackage{fixltx2e}
\usepackage{capt-of}
%\usepackage{mdframed}
\setcounter{MaxMatrixCols}{10}
\usepackage{tikz}
\usetikzlibrary{matrix}
\usepackage{endnotes}
\usepackage{soul}
\usepackage{marginnote}
%\newtheorem{theorem}{Theorem}
\newtheorem{lemma}{Lemma}
%\newtheorem{remark}{Remark}
%\newtheorem{error}{\color{Red} Error}
\newtheorem{corollary}{Corollary}
\newtheorem{proposition}{Proposition}
\newtheorem{definition}{Definition}
\newcommand{\mathsym}[1]{}
\newcommand{\unicode}[1]{}
\newcommand{\dsum} {\displaystyle\sum}
\hyphenation{op-tical net-works semi-conduc-tor}
\usepackage{pdfpages}
\usepackage{enumitem}
\usepackage{multicol}
\usepackage[utf8]{inputenc}


\headsep = 5pt
\textheight = 730pt
%\headsep = 8pt %25pt
%\textheight = 720pt %674pt
%\usepackage{geometry}

\bibliographystyle{unsrt}

\usepackage{float}

 \usepackage{xcolor}

\usepackage[framemethod=TikZ]{mdframed}
%%%%%%%FRAME%%%%%%%%%%%
\usepackage[framemethod=TikZ]{mdframed}
\usepackage{framed}
    % \BeforeBeginEnvironment{mdframed}{\begin{minipage}{\linewidth}}
     %\AfterEndEnvironment{mdframed}{\end{minipage}\par}


%	%\mdfsetup{%
%	%skipabove=20pt,
%	nobreak=true,
%	   middlelinecolor=black,
%	   middlelinewidth=1pt,
%	   backgroundcolor=purple!10,
%	   roundcorner=1pt}

\mdfsetup{%
	outerlinewidth=1,skipabove=20pt,backgroundcolor=yellow!50, outerlinecolor=black,innertopmargin=0pt,splittopskip=\topskip,skipbelow=\baselineskip, skipabove=\baselineskip,ntheorem,roundcorner=5pt}

\mdtheorem[nobreak=true,outerlinewidth=1,%leftmargin=40,rightmargin=40,
backgroundcolor=yellow!50, outerlinecolor=black,innertopmargin=0pt,splittopskip=\topskip,skipbelow=\baselineskip, skipabove=\baselineskip,ntheorem,roundcorner=5pt,font=\itshape]{result}{Result}


\mdtheorem[nobreak=true,outerlinewidth=1,%leftmargin=40,rightmargin=40,
backgroundcolor=yellow!50, outerlinecolor=black,innertopmargin=0pt,splittopskip=\topskip,skipbelow=\baselineskip, skipabove=\baselineskip,ntheorem,roundcorner=5pt,font=\itshape]{theorem}{Theorem}

\mdtheorem[nobreak=true,outerlinewidth=1,%leftmargin=40,rightmargin=40,
backgroundcolor=gray!10, outerlinecolor=black,innertopmargin=0pt,splittopskip=\topskip,skipbelow=\baselineskip, skipabove=\baselineskip,ntheorem,roundcorner=5pt,font=\itshape]{remark}{Remark}

\mdtheorem[nobreak=true,outerlinewidth=1,%leftmargin=40,rightmargin=40,
backgroundcolor=pink!30, outerlinecolor=black,innertopmargin=0pt,splittopskip=\topskip,skipbelow=\baselineskip, skipabove=\baselineskip,ntheorem,roundcorner=5pt,font=\itshape]{quaestio}{Quaestio}

\mdtheorem[nobreak=true,outerlinewidth=1,%leftmargin=40,rightmargin=40,
backgroundcolor=yellow!50, outerlinecolor=black,innertopmargin=5pt,splittopskip=\topskip,skipbelow=\baselineskip, skipabove=\baselineskip,ntheorem,roundcorner=5pt,font=\itshape]{background}{Background}

%TRYING TO INCLUDE Ppls IN TOC
\usepackage{hyperref}


\begin{document}
\title{\color{Brown}  Sugestões de Saúde Respiratória para Melhores Resultados Contra o COVID-19 \\
\vspace{-0.35ex}}
\author{Blake Elias, Chen Shen e Yaneer Bar-Yam \\ do New England Complex Systems Institute \\
\vspace{+0.35ex}
\small{\textit{(traduzido por Lucas Pontes})}\\
 \today
  \vspace{-8ex} \\
%\bigskip
\textbf{}
 }

\maketitle

%\vspace{-1ex}
%\flushbottom % Makes all text pages the same height

%\maketitle % Print the title and abstract box

%\tableofcontents % Print the contents section

\thispagestyle{empty} % Removes page numbering from the first page

%----------------------------------------------------------------------------------------
%	ARTICLE CONTENTS
%----------------------------------------------------------------------------------------

%\section*{Introduction} % The \section*{} command stops section numbering

%\addcontentsline{toc}{section}{\hspace*{-\tocsep}Introduction} % Adds this section to the table of contents with negative horizontal space equal to the indent for the numbered sections

%\tableofcontents
%\section{ Introduction}

\section*{Resumo}

O que uma pessoa pode fazer para reduzir o risco de ter um caso grave de COVID-19? Na ausência de uma cura, é importante melhorar a saúde de um indivíduo, principalmente a saúde pulmonar. Hidratação, nutrição equilibrada e sono regular podem ajudar, em conjunto com o exercício apropriado para o paciente. Uma vez infectado, recomenda-se ar fresco e limpeza do ambiente. Isso é importante para proteger aqueles que interagem com um paciente em casa ou na área da saúde. Também é consistente com a redução da exposição a partículas virais que o indivíduo pode tossir, espirrar ou expirar.

\begin{multicols}{2}
\section*{Visão geral}

A atenção ao bem-estar e ao cuidado individual durante o período moderado de COVID-19 pode afetar a probabilidade e o grau de gravidade. Entre os meios bem estabelecidos de fortalecer a resposta imune a muitos vírus estão a hidratação elevada, nutrição equilibrada (sopa de galinha ou ovo!), bons hábitos de sono e não interferir em febres (temperatura elevada), exceto se o limite de temperatura corporal de 40 graus Celsius seja ultrapassado [1] . Melhorar a saúde respiratória antes mesmo de ser infectado também deve melhorar os resultados.

É amplamente recomendada pelas autoridades de saúde manter a boa ventilação e a limpeza frequente do ambiente de indivíduos doentes com COVID-19 [2]-[5]. Isso é fundamental para qualquer pessoa que precise interagir com um paciente, sejam familiares em casa ou profissionais de saúde em ambientes médicos. Também pode fornecer benefícios através da redução da reexposição do indivíduo a partículas virais, que podem afetar o tecido pulmonar que ainda não está infectado ou que foi recentemente limpo pelo sistema imunológico.

Foi demonstrado que a respiração profunda melhora a saúde respiratória e os resultados dos pacientes em várias condições [6] - [8]. Embora uma fisioterapia respiratória mais intensiva não tenha sido considerada eficaz no tratamento de pacientes com pneumonia hospitalizada [9], os exercícios respiratórios padrão podem ser benéficos para os casos com sintomas leves.

O coronavírus se espalha através de gotículas de tosse, espirros e ar expirado de indivíduos portadores do vírus (independentemente de apresentarem sintomas). Em cerca de 80\% dos casos, o COVID-19 apresenta apenas sintomas leves e os indivíduos se recuperam sem a necessidade de intervenção médica significativa. Em 20\% dos casos, a doença se torna grave, 10\% requerem terapia intensiva para sobreviver, incluindo ventilação artificial, e 2-4\% dos casos resultam em morte. O resultado também é sensível à saúde cardiovascular subjacente, e o risco aumenta dramaticamente com a idade. Em um caso típico, a doença começa leve e, após várias semanas, de repente progride para se tornar grave. A competição entre replicação viral e eliminação pelo sistema imunológico é o que ocorre "nos bastidores" da progressão ou não da doença. Um início repentino de gravidade indica que a batalha atingiu uma transição (ponto de inflexão) para uma fase diferente. Isso pode ser devido à extensão de danos no tecido pulmonar; sobrecarga de alguma capacidade do sistema imunológico; impactos autoimunes, como uma tempestade de citocinas; ou outros mecanismos. A sensibilidade da transição para múltiplos fatores sugere que mesmo uma pequena mudança nas condições individuais pode mudar o equilíbrio. Fortalecer o sistema imunológico ou reduzir a capacidade do vírus se espalhar pelo tecido pulmonar pode ser útil.

Queremos deixar claro três coisas: (1) as recomendações a seguir para melhorar a saúde pulmonar e reduzir a exposição/reexposição a partículas virais \textbf{são seguras apenas para indivíduos com saúde geral razoável}. Qualquer pessoa com problemas específicos de saúde ou problemas respiratórios deve consultar seu médico antes de adotá-las; (2) observe que essas recomendações \textbf{não substituem a prevenção contra a infecção por COVID-19}; (3) esperamos que essas recomendações reduzam a gravidade \textbf{apenas em alguns casos}.

\section*{Recomendações}

\begin{itemize}
\item \textbf{Exercício aeróbico}. Antes da infecção, recomenda-se exercícios aeróbicos para fortalecer a saúde cardiovascular. Uma vez infectado, durante o período de sintomas leves, o exercício aeróbico diário leve pode melhorar a ventilação pulmonar, e também pode beneficiar a função imune [10]. Idealmente, faça este exercício ao ar livre ou com janelas abertas ou áreas bem ventiladas. Em climas suficientemente quentes, caminhadas mais longas ou até corridas podem melhorar a capacidade pulmonar. Saltar polichinelos, correr no lugar ou dançar podem ser feito mesmo em espaços pequenos.
\item \textbf{Mantenha as janelas abertas onde as temperaturas permitirem}. É melhor que os fluxos de ar sejam externos, e que certamente não permitam que o fluxo de ar de um indivíduo infectado vá para espaços onde indivíduos não infectados estão presentes [5]. Isso tem dois benefícios: (1) permitir que quaisquer partículas virais presentes no ar saiam da sala, em vez de você (ou outra pessoa) respirá-las novamente; (2) trazer mais oxigênio para a sala - útil para os pulmões e para a saúde geral. Se o clima na sua região estiver frio, considere abrir a janela pelo menos um pouco enquanto um aquecedor também esteja ligado. Purificadores de ar também podem ser úteis.
\item \textbf{Passe algum tempo ao ar livre (sempre que o clima estiver ameno, e sem se aproximar de outras pessoas dentro de um raio de 6 pés [11])}. Varandas, quintais ou pátios são bons locais para se estar, além de caminhadas, evitando a proximidade com outras pessoas. Isso tem os mesmos benefícios que manter as janelas abertas - garante que as partículas virais exaladas não sejam inaladas novamente.
\item \textbf{Inspire o ar pelo nariz}. Respirar pelo nariz ajuda a limpar o ar recebido, através dos cílios (pêlos pequenos) e membranas mucosas, criando assim um escudo contra doenças. A respiração nasal também aquece e umedece o ar que entra.
\item \textbf{Respiração profunda}. A inspiração e a expiração profundas trazem ar fresco e podem melhorar a capacidade pulmonar. Normalmente, inspiramos e expiramos apenas uma fração da capacidade de nossos pulmões. A expulsão de partículas virais das áreas mais estagnadas do pulmão pode diminuir ainda mais a auto-exposição a partículas virais. A respiração profunda é frequentemente recomendada para a saúde e o bem-estar, e pode ser feita várias vezes ao dia em um horário regular.
\item \textbf{Práticas adicionais de saúde pulmonar}. Muitos exercícios adicionais podem ser encontrados para melhorar a saúde respiratória. Veja as recomendações do Rush University Medical Center [12] para exercícios mais diferenciados.
\item textbf{Limpe as superfícies, e lave roupas e roupas de cama}. A lavagem frequente remove as partículas virais que se depositam nas superfícies e nas roupas, e evita a exposição ou reexposição a essas partículas.
\end{itemize}

\end{multicols}


\section*{Referências}

[1] Sharon S Evans, Elizabeth A Repasky, and Daniel T Fisher. Fever and the thermal regulation of immunity: the immune system feels the heat. Nature Reviews Immunology, 15(6):335–349, 2015.

[2] US Centers for Disease Control and Prevention. Preventing the Spread of Coronavirus Disease 2019 in Homes and Residential Communities, February 14, 2020. "https://www.cdc.gov/coronavirus/2019-ncov/hcp/guidance-prevent-spread.html".

[3] New Zealand Ministry of Health. COVID-19 (novel coronavirus) – staying at home (self-isolation), March 15, 2020. "https: //www.health.govt.nz/our-work/diseases-and-conditions/covid-19-novel-coronavirus/covid-19-novel-coronavirus-health-advice-general-public/covid-19-novel-coronavirus-staying-home-self-isolation.

[4] Public Health Agency of Canada. Community-based measures to mitigate the spread of coronavirus disease (COVID-19) in Canada, March 12, 2020. https://www.canada.ca/en/public-health/services/diseases/2019-novel-coronavirus-infection/health-professionals/public-health-measures-mitigate-covid-19.html.

[5] STCN. [Anti-epidemic Science] Doctor Zhang Wenhong calls you to open the window! Accept this “ventilation timetable”, March 9, 2020. http://news.stcn.com/2020/0309/15711677.shtml.

[6] M Vitacca, Enrico Clini, L Bianchi, and N Ambrosino. Acute effects of deep diaphragmatic breathing in copd patients with chronic respiratory insufficiency. European Respiratory Journal, 11(2):408–415, 1998.

[7] Elisabeth Westerdahl, Anna Wittrin, Margareta Kånåhols, Martin Gunnarsson, and Ylva Nilsagård. Deep breathing exercises with positive expiratory pressure in patients with multiple sclerosis – a randomized controlled trial. The clinical respiratory journal, 10(6):698–706, 2016.

[8] Elisabeth Westerdahl. Optimal technique for deep breathing exercises after cardiac surgery. Minerva Anestesiol, 81(6):678–683, 2015.

[9] Sven Britton, Margareta Bejstedt, and Lars Vedin. Chest physiotherapy in primary pneumonia. Br Med J (Clin Res Ed), 290(6483):1703–1704, 1985.

[10] Medline Plus. Exercise and immunity, accessed March 15, 2020. https://medlineplus.gov/ency/article/007165.htm.

[11] US Centers for Disease Control. Interim US Guidance for Risk Assessment and Public Health Management of Persons with Potential Coronavirus Disease 2019 (COVID-19) Exposures: Geographic Risk and Contacts of Laboratory-confirmed Cases, March 9, 2020. https://www.cdc.gov/coronavirus/2019-ncov/php/risk-assessment.html.

[12] Rush University Medical Center. 8 Tips for Healthy Lungs, accessed March 15, 2020. https://www.rush.edu/health-wellness/discover-health/8-tips-healthy-lungs.

% \bibliography{MyCollection.bib}
\bibliography{references.bib}

\end{document}

